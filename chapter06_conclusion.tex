\chapter{Conclusion}

In this thesis, an end-to-end neural network for autonomous driving with basic framework
The ROS 2 is a flexible framework which enables fast communication between components. The version of rosbridge used for this project, ros2-web-bridge, carries too much overhead and affects the performance. It is advised to use the new lgsvl-ros2-bridge from the LGSVL team instead.
LGSVL simulator provides a great solution to collect and test algorithms. In about 24 hours, it is possible to collect more than 250,000 data entries and label them.
is implemented. Upon evaluating, we observe the following:
\begin{enumerate}
    \item Light plays a major role on how good the model predict the correct decision as
        shown in \ref{chapter05subsec:setup1}.
    \item Steering performs remarkably well even when trained just with RGB-Grayscale image.
    \item Acceleration and braking work best when performed as classification tasks.
    \item First dataset (dataset \ref{chapter05list:ds1}) though was collected without traffic,
        performs excellently when exposed to it. It also deals with dynamic and static objects which signifies the generalisation example of supervised learing.
    \item All the datasets(\ref{fig:datasetscomparectrlcmds}) are imbalanced with
        \textit{no action} state taking majority of the entries. This causes a bias in
        decision making.
    \item Splitting the fully connected layers help in optimising the weights calculations
        resulting in somewhat balanced prediction.
    \item Including a highly varying auxiliary task such as velocity skews the model to
        predict it than its primary task. Therefore weighting the cost function is most
        useful in suppressing the velocity and enhancing the primary task.
    \item Adjusting the width and depth of the convolutional layers is almost as difficult
        as adjusting the fully connected layer. Increase in trainable parameter doesn't
        really mean more features.
    \item Both depth and segmented images carry no feature information such as lane
        markings which make it harder to predict outputs using only them. They are,
        however, highly useful when combined with a RGB-Grayscale image.
    \item Late fusion takes longer than early fusion technique as the fusing operation is
        done while training the model.
    \item RGB-Grayscale plus segmented images fusion give much better results and perform really well even at night and wet conditions.
    \item While evaluation, bigger and more complex the prediction NN model, longer it
        takes to predict. This will then delay the output being published to the
        simulator.
\end{enumerate}

To conclude, this project with the help of docker, ROS, LGSVL, Keras, Tensorflow, and Python, it is possible to construct a basic end-to-end model that incorporates state of the art data fusion technique to help make better decisions. However, without regular feedback by testing the algorithm from real-world conditions, significant progress in this field would be difficult. 