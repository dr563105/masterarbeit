\chapter{Appendix}
\begin{table}[!ht]
    \centering
\begin{tabular}{lc}
    \toprule
    Criteria(Softmax/Categorical crossentropy) & Dataset 3 \\\midrule
    Lane keeping/Drive straight  & 4  \\
    Gradual acceleration increase  & 3 \\
    Smooth braking behaviour observed & 3 \\
    Smooth steering control at high speed(10m/s) & 2 \\
    Smooth steering control at turnings & 1\\
    Avoids colliding into static objects & 1 \\
    Detects vehicles as dynamic objects & 5 \\
    Navigates traffic smoothly & 3\\
    No random braking & 1 \\\bottomrule
\end{tabular}
\caption{Separate dense layers - How the model evaluates to different criteria. Rating 1
to 5. 1 being the lowest.}
\label{table:ccedense}
\end{table}
\begin{figure}[!ht]
    \centering
    \def\svgwidth{0.8\textwidth}
    \input{figures/inkscape/ccecompareRadar2.pdf_tex}
    \caption{Categorical cross entropy dataset 3 evaluation comparison across neural
    network changes - Basic and split at dense layers models. Template from soccerplots\cite{soccerplots}}
    \label{fig:radarccecompare1}
\end{figure}
\begin{table}[!ht]
    \centering
\begin{tabular}{lc}
    \toprule
    Criteria(Softmax/Categorical crossentropy) & Dataset 3 \\\midrule
    Lane keeping/Drive straight  & 4  \\
    Gradual acceleration increase  & 3\\
    Smooth braking behaviour observed & 3 \\
    Smooth steering control at high speed(10m/s) & 4 \\
    Smooth steering control at turnings & 1\\
    Avoids colliding with static objects & 2 \\
    Detects vehicles as dynamic objects & 5 \\
    Navigates traffic smoothly & 4\\
    No random braking & 1 \\
    Smooth evaluation experience & 3 \\\bottomrule

\end{tabular}
\caption{Split at the LSTM layer - How the model evaluates to different criteria. Rating 1
to 5. 1 being the lowest.}
\label{table:cceLSTM}
\end{table}
\begin{table}[!ht]
    \centering
\begin{tabular}{lc}
    \toprule
    Criteria(Softmax/Categorical crossentropy) & Dataset 3 \\\midrule
    Lane keeping/Drive straight  & 4  \\
    Gradual acceleration increase  & 4\\
    Smooth braking behaviour observed & 4 \\
    Smooth steering control at high speed(10m/s) & 4 \\
    Smooth steering control at turnings & 2\\
    Avoids colliding with static objects & 3 \\
    Detects vehicles as dynamic objects & 5 \\
    Navigates traffic smoothly & 3\\
    No random braking & 1 \\
    Smooth evaluation experience & 4 \\\bottomrule
\end{tabular}
\caption{Separate neural network for classification and steering outputs - How the model
evaluates to different criteria. Rating 1 to 5. 1 being the lowest.}
\label{table:cce2NN}
\end{table}
\begin{figure}[!ht]
    \centering
    \def\svgwidth{0.8\textwidth}
    \input{figures/inkscape/ccecompareRadar1.pdf_tex}
    \caption{Categorical cross entropy dataset 3 evaluation comparison across neural
    network changes - Separate LSTM layers and separate neural network models. Template from soccerplots\cite{soccerplots}}
    \label{fig:radarccecompare2}
\end{figure}
\begin{table}[!ht]
    \centering
\begin{tabular}{lc}
    \toprule
    Criteria & Rating \\\midrule
    Lane keeping/Drive straight  & 5  \\
    Gradual acceleration increase  & 3\\
    Smooth braking behaviour observed & 2 \\
    Smooth steering control at high speed(10m/s) & 2\\
    Smooth steering control at turnings & \textbf{5}\\
    Avoids colliding with static objects & \textbf{5} \\
    Detects vehicles as dynamic objects & 4\\
    Navigates traffic smoothly & 4\\
    Sunlight glare & \textbf{4} \\
    Low-light or night driving & \textbf{4} \\
    No random braking & 1 \\
    Evaluation experience & 5 \\\bottomrule
\end{tabular}
\caption{Larger dataset early fusion evaluation - RGB-Grayscale and Segmented images.
Rating 1 to 5. 1 being the lowest.}
\label{table:earlyfusionrgbseg200k}
\end{table}
\begin{figure}[!ht]
    \centering
    \def\svgwidth{0.5\textwidth}
    \input{figures/inkscape/nightdriving.pdf_tex}
    \caption{Night time driving with RGB-G and Segmented images fusion - Traffic vs
    collisions}
    \label{fig:224kdatafusionnight}
\end{figure}

