\chapter{Fundamentals}
\section{Machine learning: What and why?}
Machine learning is all about of learning from data; gaining knowledge from it. More the
data, more the chances to learn. Machine learning was initially thought of as automating
redundant human tasks and later developed into something that allowed solving complex
mathematical problems. It was seen as an addition to humans than extension of them.
Machine learning these days are required to perform tasks that are quite obvious and
natural to humans; such as recognising faces in images or perceive the road, environment
around the vehicle and make decisions instinctively.

All these attributes require to extend the field of machine learning to extract more
knowledge from a given data. The figure \ref{fig:ai_ml_dl} shows how artificial
intelligence(AI) is divided to specific areas -- Machine learning(ML) and deep learning(DL).

So, in this chapter, an overview is given on the concepts that are used in the later chapters.

\begin{figure}[h]
	\centering
    \def\svgwidth{0.5\textwidth}
    \input{figures/inkscape/aimldl.pdf_tex} %use full path to know the location of pdftex
    \caption{Schema of AI, ML and DL}
    \label{fig:ai_ml_dl}
\end{figure}

\subsection{Learning algorithms}
Machine learning provides a means to tackle tasks that are complex to solve through fixed
programmes and designed by human beings \cite{Goodfellow-et-al-2016}. A learning algorithm
is an algorithm which gains the ability to learn from data. A ML algorithm is one that
gains the ability to learn from an experience E with respect to some class of tasks T and
performance measure P \cite{mitchell1996m}.

\subsubsection*{Tasks, T}
The two major tasks in ML are \textit{classification} and \textit{regression}.

In classification related tasks, the system is identify which of \textit{k} categories an
input belongs to. A function $f : \mathbb{R}^n \rightarrow\{1, \ldots,k\}$ is used by the
learning algorithm to solve this task. When $y = f(x)$, the model assigns an input
described by vector $\mathbf{x}$ to a category
identified by numeric code $y$. There are other variants of the classification
task, for example, where $f$ outputs a probability distribution over classes
\cite{Goodfellow-et-al-2016_1}. Alexnet \cite{Alexnet2012} is one of the examples of
classification tasks that performed object recognition.

Regression is similar to classification except that the output is a continuous value. A function $f
": \mathbb{R}^n \rightarrow \mathbb{R}$ predicts a numerical value for some input.
Predicting the steering control value is a prime example for regression task.

There are ofcourse other tasks but only classification and regession are used in this
thesis. Hence the narrow focus.

\subsubsection*{Performance measure, P}
To evaluate the performance of a ML algorithm, it is must to design quantitative measure
of its performance. Usually this performance measure P is specific to the task T. There
are two distinct types of measurements -- accuracy and error rate.

If the goal is to learn a mapping from inputs $x$ to outputs $y$, where $y \in \{1,\ldots
, C\}$, with $C$ being the number of classes. If $C = 2$, this is
called binary classification (in which case we often assume $y \in \{0, 1\})$; if $C > 2$, this is called
multiclass classification. If the class labels are not mutually exclusive (e.g., somebody may be
classified as tall and strong), we call it multi-label classification
\cite{murphy2013machine_1}.

Accuracy is a proportion of how much the model produces the correct outputs.
So in the case of binary classification, if the function $f$ predicts a probability
densities $\hat y \in \{0.3, 0.7\}$, for a ground truth $y$ of value $1$, then P is $70\%$
accurate or the error rate is $30\%$.

It is must that the model is evaluated with a data that it has not seen before. This data
\textit{testing set}, gives a good judgement on the performance.

\subsubsection*{Experience, E}
The ML algorithms can be classified into \textit{supervised}, \textit{unsupervised} and
\textit{reinforcement} learning based on the kind of experience they are allowed to have.
A learning algorithm is allowed to gain experience by going through the \textit{dataset}.
A dataset is collection of all the examples for a given task. For example, to classify
which category a shown image belongs to has collection of images as dataset
\cite{cifar10}. Sometimes datasets are also called as \textit{data points}.

The focus will be on supervised learning in our case. The CIFAR
dataset \cite{cifar10} containing images as features has \textit{targets} or
\textit{labels} associated with it. In supervised learning(SL), the target functionality
is shown to the learning algorithm. A random vector $\mathbf{x}$ explicitly attempts to
learn the probability distribution $p(\mathbf{x})$ and predicts $\mathbf{y}$ from
$\mathbf{x}$, usually estimating $p(\mathbf{y}\mid\mathbf{x})$.

\section{Deep Learning}
Deep learning is a subset of machine learning. It takes all the algorithms, concepts from
the machine learning and narrows the focus to enable a model learn from data to do tasks
that involve less human interaction, large number of data and parameters.

\subsection{Simple neural network}
\textit{Linear regression} is one of the common SL algorithms. It solves the regression
problem. For example, if there is vector $\mathbf{x} \in \mathbb{R}^n$ as input and
predict a scalar value $y \in \mathbb{R}$ as its output, then in linear regression, output
is a linear function of the input. We can define it as
\begin{equation}
    \hat y = \mathbf{w}^T\mathbf{x}
\end{equation}
where $\mathbf{w} \in \mathbb{R}^n$ is a vector of parameters.

$\mathbf{w}$ is usually referred to as a set of weights that determine how each feature
affects the prediction. A $\mathbf{w}_i$ is simply multiplied with a feature $x_i$ to
predict $\hat y$. By manipulating the $\mathbf{w}_i$ value, the corresponding feature has
an effect on the prediction  $\hat y$.

A learning algorithm, in this case linear regression, is implemented as a perceptron. It
is a single-layer neural network. They generally consists of four main parts -- input
nodes $x_i$, weights $w_i$, bias $b_0$(if necessary), net sum $\Sigma$ and an activation
function $\sigma$. This is shown in the figure \ref{fig:simpleNN}.

\begin{figure}[h]
    \centering
        \def\svgwidth{0.5\textwidth}
        \documentclass[crop, tikz]{standalone}
\usepackage{tikz}

\usetikzlibrary{positioning}

\tikzstyle{inputNode}=[draw,circle,minimum size=10pt,inner sep=0pt]
\tikzstyle{stateTransition}=[-stealth, thick]

\begin{document}
\begin{tikzpicture}
	\node[draw,circle,minimum size=25pt,inner sep=0pt] (x) at (0,0) {$\Sigma$ $\sigma$};

	\node[inputNode] (x0) at (0, 1.75) {$\tiny +1$};
	\node[inputNode] (x1) at (-2, 0.75) {$\tiny x_1$};
	\node[inputNode] (x2) at (-2, 0) {$\tiny x_2$};
	\node[inputNode] (x3) at (-2, -0.75) {$\tiny x_3$};
	\node[inputNode] (xn) at (-2, -1.75) {$\tiny x_n$};

	\draw[stateTransition] (x0) to[out=-90,in=90] node [midway, right] {$b_0$} (x);
	\draw[stateTransition] (x1) to[out=0,in=150] node [midway, sloped, above] {$w_1$} (x);
	\draw[stateTransition] (x2) to[out=0,in=180] node [midway, sloped, above] {$w_2$} (x);
	\draw[stateTransition] (x3) to[out=0,in=210] node [midway, sloped, above] {$w_3$} (x);
	\draw[stateTransition] (xn) to[out=0,in=240] node [midway, sloped, above] {$w_n$} (x);
	\draw[stateTransition] (x) -- (4.2,0) node [midway,above] {$\sigma\left(w_0 + \sum\limits_{i=1}^{n}{w_ix_i}\right)$};
	\draw[dashed] (0,-0.43) -- (0,0.43);
	\node (dots) at (-2, -1.15) {$\vdots$};
	\node[inputNode, thick] (i1) at (6, 0.75) {$\tiny i_1$};
	\node[inputNode, thick] (i2) at (6, 0) {$\tiny i_2$};
	\node[inputNode, thick] (i3) at (6, -0.75) {$\tiny i_3$};
	
	\node[inputNode, thick] (h1) at (8, 1.5) {$\tiny h_1$};
	\node[inputNode, thick] (h2) at (8, 0.75) {$\tiny h_2$};
	\node[inputNode, thick] (h3) at (8, 0) {$\tiny h_3$};
	\node[inputNode, thick] (h4) at (8, -0.75) {$\tiny h_4$};
	\node[inputNode, thick] (h5) at (8, -1.5) {$\tiny h_5$};
	
	\node[inputNode, thick] (o1) at (10, 0.75) {$\tiny o_1$};
	\node[inputNode, thick] (o2) at (10, -0.75) {$\tiny o_2$};
	
	\draw[stateTransition] (5, 0.75) -- node[above] {$I_1$} (i1);
	\draw[stateTransition] (5, 0) -- node[above] {$I_2$} (i2);
	\draw[stateTransition] (5, -0.75) -- node[above] {$I_3$} (i3);
	
	\draw[stateTransition] (i1) -- (h1);
	\draw[stateTransition] (i1) -- (h2);
	\draw[stateTransition] (i1) -- (h3);
	\draw[stateTransition] (i1) -- (h4);
	\draw[stateTransition] (i1) -- (h5);
	\draw[stateTransition] (i2) -- (h1);
	\draw[stateTransition] (i2) -- (h2);
	\draw[stateTransition] (i2) -- (h3);
	\draw[stateTransition] (i2) -- (h4);
	\draw[stateTransition] (i2) -- (h5);
	\draw[stateTransition] (i3) -- (h1);
	\draw[stateTransition] (i3) -- (h2);
	\draw[stateTransition] (i3) -- (h3);
	\draw[stateTransition] (i3) -- (h4);
	\draw[stateTransition] (i3) -- (h5);
	
	\draw[stateTransition] (h1) -- (o1);
	\draw[stateTransition] (h1) -- (o2);
	\draw[stateTransition] (h2) -- (o1);
	\draw[stateTransition] (h2) -- (o2);
	\draw[stateTransition] (h3) -- (o1);
	\draw[stateTransition] (h3) -- (o2);
	\draw[stateTransition] (h4) -- (o1);
	\draw[stateTransition] (h4) -- (o2);
	\draw[stateTransition] (h5) -- (o1);
	\draw[stateTransition] (h5) -- (o2);
	
	\node[above=of i1, align=center] (l1) {Input \\ layer};
	\node[right=2.3em of l1, align=center] (l2) {Hidden \\ layer};
	\node[right=2.3em of l2, align=center] (l3) {Output \\ layer};
	
	\draw[stateTransition] (o1) -- node[above] {$O_1$} (11, 0.75);
	\draw[stateTransition] (o2) -- node[above] {$O_2$} (11, -0.75);
	
	\path[dashed, double, ultra thick, gray] (x.north) edge[bend left=0] (h5.north);
	\path[dashed, double, ultra thick, gray] (x.south) edge[bend right=0] (h5.south);
\end{tikzpicture}
\end{document}

        \caption{A simple neutral network}
        \label{fig:simpleNN}
\end{figure}

\subsubsection*{Activation function}
The common activation functions used are Rectified Linear unit(ReLu), Sigmoid, tanh and
softmax function. For each type of activation, $\sigma$ then decides if the input received is
relevant or not relevant. To convert linear inputs to non-linear, all that has to be done
is to use a non-linear activation function. In the figure \ref{fig:activationfunctions},
shows the characteristics of some of the activation functions.
\begin{figure}[ht]
	\begin{center}
   \def\svgwidth{0.5\textwidth}
    \input{figures/inkscape/activationfn.pdf_tex} %use full path to know the location of pdftex
	\end{center}
    \caption{Activation functions}
    \label{fig:activationfunctions}
\end{figure}
For classification tasks, usually the last layer of the networks is equipped with softmax
activation layer. This function normalises the output to a probability distribution over
predicted output classes.

\subsubsection*{Multilayer feedforward networks}
Deep feedforward networks or multilayer perceptrons are the quintessential deep learning
models. Its goal is to approximate function $f^*$. In the below figure \ref{fig:MLP},
information flows from inputs $\mathbf{x}$ to output $y$ using a mapping function
$\mathbf{y} = f(\mathbf{x};\mathbf{\theta})$ where $\theta$ are the parameters values
which the MLP learns for optimal approximation.

They are called feedforward as there are no feedback connections in which outputs of the
model are fed back into itself. Feedforwards networks with feedbacks are called
\textit{recurrent neutral networks}.

Feedforwards networks form the core for many commercial applications. For example, the
convolutional neural networks used for object detection are a special kind of feedforward
networks.
\begin{figure}[h]
    \def\svgwidth{0.5\textwidth}
	\begin{center}
        \input{figures/inkscape/inputhiddenoutput.pdf_tex} %use
    \end{center}
    \caption{Multi layer perceptrons}
    \label{fig:MLP}
\end{figure}

More the hidden layers, more the depth of the feedforward networks. The width is given by
the dimensionality of the hidden layer.

\subsubsection*{Loss function}
As mentioned before, a mapping function $f$ noisily approximates the input $x$ to output
$y$. So, the noise or the deviation from the true value(ground truth) must be kept at
minimum. The function that calculates the deviation is called \textit{cost} or
\textit{loss} function. It is important to choose the right loss function for a model.

\begin{figure}[h]
	\centering
    \def\svgwidth{0.5\textwidth}
    \input{figures/inkscape/mse.pdf_tex} %use full path to know the location of pdftex
    \caption{Mapping from x to y. The predictor is shown as linear line. The distance
    between the true values and predictor gives the loss. The sum of all the distances
gives the loss function.}
\label{fig:loss function}
\end{figure}

For multi-label classification tasks, \textit{categorical cross-entropy} function is used.
For each category, cross-entropy is calculated. The difference between the cross-entropy
of training data and the model's predictions is the cost function.

For regression tasks, the models are subjected to loss functions such as \textit{mean
absolute error}(MAE), \textit{mean squared error}(MSE) and \textit{mean squared
logarithmic error}(MSLE). In MAE, the mean of absolute differences among predictions and
expected results are calculated.
\begin{equation}
    MAE = \frac{1}{n}\sum_{i=1}^n\left |y_i -\hat y_i \right|
\end{equation}
In MSE, the mean of squared differences among predictions and true outputs are
calculated.
\begin{equation}
    MSE = \frac{1}{n}\sum_{i=1}^n (y_i - \hat y_i)^2
\end{equation}
In MSLE, the mean of relative distances between predictions and true outputs are
calculated.
\begin{equation}
    MSLE = \frac{1}{n}\sum_{i=1}^n(log(y_i+1)-log(\hat y_i+1))^2
\end{equation}

\subsubsection*{Gradient descent} \label{gradientdescent}
Gradient descent is another technique to minimise the cost function parameterised by a
model parameter $\mathbf{w}$. The first derivative(or gradient) gives the slope of the
cost function. Hence, to minimise it, direction opposite to the gradient is chosen.

\begin{figure}[h]
	\centering
    \def\svgwidth{0.5\textwidth}
        \input{figures/inkscape/minima.pdf_tex}
    \caption{Finding the stochastic gradient descent}
    \label{fig:gradientdescent}
\end{figure}

The rate at the which the gradient step reduces is given by the \textit{learning rate}. If
the learning rate is high, greater the step size of each gradient; possibly causing the
step to miss the global minima. Lower the learning rate, more steps or training cycles
needed to reach the global minima. So the rate must be carefully chosen for each model.

\subsubsection*{Backpropagation}
Backpropagation are a class of algorithms which help in training feedforward neural
networks for supervised learning. A model is said to fit when the gradient computation of
the loss function is efficient w.r.t the weights of single input-output in the network.
Backpropagation performs the effective gradient computation using the loss functions
explained above.

\subsubsection*{Optimizer}
The loss function was able to explain how far the predictions were compared to the true
outputs in a mathematical way. During training process, certain parameters can be tweaked
to help the loss function predict correct and optimised results. However, there are
question such as how to change them, by how much and when?

This is exactly optmizer's function. As explained in \ref{gradientdescent }, gradient
descent and learning rate form the core of optimizer's functionality. \textit{Stocastic
gradient descent}(SGD) is one of the oldest techniques in which gradients for all of your
training examples on every pass of gradient descent are calculated. However, they are slow
and require much computation power. Some of the other popular optimizers are Adam,
Adaguard, RMSprop. In this work, Adam is used. Adam stands for adaptive moment estimation.
It is a combination of all the advantages of two other extensions of SGD -- Adaguard and
RMSprop. Adam is computationally efficient, straight forward to implement, invariant to
diagonal rescale of the gradients, less effort need to hyperparameters tuning.

\subsubsection*{Challenges in Machine learning algorithms}
\begin{enumerate}
    \item insufficient labelled data
    \item poor quality data and irrelevant features
    \item overfitting/underfitting the model
\end{enumerate}

The first two issues can be solved if the user is careful during data collection and does
preprocessing before feeding the data to the training model. However, if the training or
the test data is too small, the model is subject to underfitting or overfitting. Though
our aim is to reduce the error in the training set, we also need to reduce the error in
the test set. The gap between training and testing error is also important parameter.
Underfitting occurs when the model is not able to obtain sufficiently low error value for
the training set. And if the gap between training and testing error is too large,
overfitting happens. The sweet spot is to stop training the model when the gap between the
two sets is at a minimum value. Left of the optimal point, the model underfits. Right of
it, the model overfits. The below figure \ref{fig:overfittingunderfitting} shows it very
well. Validation error is the error calculated for the test set.
\begin{figure}[h]
	\centering
    \def\svgwidth{0.75\textwidth}
    \input{figures/inkscape/overfitting.pdf_tex} %use full path to know the location of pdftex
    \caption{Relationship between capacity and error. Inspired from
    \cite{Goodfellow-et-al-2016}}
    \label{fig:overfittingunderfitting}
\end{figure}


\begin{figure}[h]
	\centering
    \def\svgwidth{0.5\textwidth}
   % \begin{Large}
    \documentclass[crop, tikz]{standalone}
\usepackage{tikz}
\usepackage{bm}
\usepackage{relsize}

\usetikzlibrary{positioning}

\begin{document}
\begin{tikzpicture}

	\node[circle, draw, thick] (i1) {};
	\node[circle, draw, thick, above=2em of i1] (i2) {};
	\node[circle, draw, thick, above=2em of i2] (i3) {};
	\node[circle, draw, thick, below=2em of i1] (i4) {};
	\node[circle, draw, thick, below=2em of i4] (i5) {};
	
	\node[circle, draw, thick, right=4em of i1] (h1) {};
	\node[circle, draw, thick, right=4em of i2] (h2) {};
	\node[circle, draw, thick, right=4em of i3] (h3) {};
	\node[circle, draw, thick, right=4em of i4] (h4) {};
	\node[circle, draw, thick, right=4em of i5] (h5) {};
	
	\node[circle, draw, thick, right=4em of h1] (hh1) {};
	\node[circle, draw, thick, right=4em of h2] (hh2) {};
	\node[circle, draw, thick, right=4em of h3] (hh3) {};
	\node[circle, draw, thick, right=4em of h4] (hh4) {};
	\node[circle, draw, thick, right=4em of h5] (hh5) {};
	
	\node[circle, draw, thick, right=4em of hh2] (o1) {};
	\node[circle, draw, thick, right=4em of hh4] (o2) {};
	
	\draw[-stealth, thick] (i1) -- (h1);
	\draw[-stealth, thick] (i1) -- (h2);
	\draw[-stealth, thick] (i1) -- (h3);
	\draw[-stealth, thick] (i1) -- (h4);
	\draw[-stealth, thick] (i1) -- (h5);
	\draw[-stealth, thick] (i2) -- (h1);
	\draw[-stealth, thick] (i2) -- (h2);
	\draw[-stealth, thick] (i2) -- (h3);
	\draw[-stealth, thick] (i2) -- (h4);
	\draw[-stealth, thick] (i2) -- (h5);
	\draw[-stealth, thick] (i3) -- (h1);
	\draw[-stealth, thick] (i3) -- (h2);
	\draw[-stealth, thick] (i3) -- (h3);
	\draw[-stealth, thick] (i3) -- (h4);
	\draw[-stealth, thick] (i3) -- (h5);
	\draw[-stealth, thick] (i4) -- (h1);
	\draw[-stealth, thick] (i4) -- (h2);
	\draw[-stealth, thick] (i4) -- (h3);
	\draw[-stealth, thick] (i4) -- (h4);
	\draw[-stealth, thick] (i4) -- (h5);
	\draw[-stealth, thick] (i5) -- (h1);
	\draw[-stealth, thick] (i5) -- (h2);
	\draw[-stealth, thick] (i5) -- (h3);
	\draw[-stealth, thick] (i5) -- (h4);
	\draw[-stealth, thick] (i5) -- (h5);
	
	\draw[-stealth, thick] (h1) -- (hh1);
	\draw[-stealth, thick] (h1) -- (hh2);
	\draw[-stealth, thick] (h1) -- (hh3);
	\draw[-stealth, thick] (h1) -- (hh4);
	\draw[-stealth, thick] (h1) -- (hh5);
	\draw[-stealth, thick] (h2) -- (hh1);
	\draw[-stealth, thick] (h2) -- (hh2);
	\draw[-stealth, thick] (h2) -- (hh3);
	\draw[-stealth, thick] (h2) -- (hh4);
	\draw[-stealth, thick] (h2) -- (hh5);
	\draw[-stealth, thick] (h3) -- (hh1);
	\draw[-stealth, thick] (h3) -- (hh2);
	\draw[-stealth, thick] (h3) -- (hh3);
	\draw[-stealth, thick] (h3) -- (hh4);
	\draw[-stealth, thick] (h3) -- (hh5);
	\draw[-stealth, thick] (h4) -- (hh1);
	\draw[-stealth, thick] (h4) -- (hh2);
	\draw[-stealth, thick] (h4) -- (hh3);
	\draw[-stealth, thick] (h4) -- (hh4);
	\draw[-stealth, thick] (h4) -- (hh5);
	\draw[-stealth, thick] (h5) -- (hh1);
	\draw[-stealth, thick] (h5) -- (hh2);
	\draw[-stealth, thick] (h5) -- (hh3);
	\draw[-stealth, thick] (h5) -- (hh4);
	\draw[-stealth, thick] (h5) -- (hh5);
	
	
	\draw[-stealth, thick] (hh1) -- (o1);
	\draw[-stealth, thick] (hh1) -- (o2);
	\draw[-stealth, thick] (hh2) -- (o1);
	\draw[-stealth, thick] (hh2) -- (o2);
	\draw[-stealth, thick] (hh3) -- (o1);
	\draw[-stealth, thick] (hh3) -- (o2);
	\draw[-stealth, thick] (hh4) -- (o1);
	\draw[-stealth, thick] (hh4) -- (o2);
	\draw[-stealth, thick] (hh5) -- (o1);
	\draw[-stealth, thick] (hh5) -- (o2);
	
	\draw[-stealth, double, dashed, thick] (5.5,0) -- node[above] {dropout} (8.6, 0);
	
	
	%%% BOUNDARY %%%
	
	\node[circle, draw, thick, red, fill=red!10, right=15em of hh1] (i1) {};
	\node[circle, draw, thick, red, fill=red!10, above=2em of i1] (i2) {};
	\node[circle, draw, thick, above=2em of i2] (i3) {};
	\node[circle, draw, thick, below=2em of i1] (i4) {};
	\node[circle, draw, thick, below=2em of i4] (i5) {};
	
	\node[red] (icr) at (i1) {$\mathlarger{\mathlarger{\mathlarger{\mathlarger{\mathlarger{\bm{\times}}}}}}$};
	\node[red] (icr) at (i2) {$\mathlarger{\mathlarger{\mathlarger{\mathlarger{\mathlarger{\bm{\times}}}}}}$};
	
	\node[circle, draw, thick, red, fill=red!10, right=4em of i1] (h1) {};
	\node[circle, draw, thick, right=4em of i2] (h2) {};
	\node[circle, draw, thick, red, fill=red!10, right=4em of i3] (h3) {};
	\node[circle, draw, thick, red, fill=red!10, right=4em of i4] (h4) {};
	\node[circle, draw, thick, right=4em of i5] (h5) {};
	
	\node[red] (icr) at (h1) {$\mathlarger{\mathlarger{\mathlarger{\mathlarger{\mathlarger{\bm{\times}}}}}}$};
	\node[red] (icr) at (h3) {$\mathlarger{\mathlarger{\mathlarger{\mathlarger{\mathlarger{\bm{\times}}}}}}$};
	\node[red] (icr) at (h4) {$\mathlarger{\mathlarger{\mathlarger{\mathlarger{\mathlarger{\bm{\times}}}}}}$};
	
	\node[circle, draw, thick, right=4em of h1] (hh1) {};
	\node[circle, draw, thick, red, fill=red!10, right=4em of h2] (hh2) {};
	\node[circle, draw, thick, right=4em of h3] (hh3) {};
	\node[circle, draw, thick, red, fill=red!10, right=4em of h4] (hh4) {};
	\node[circle, draw, thick, right=4em of h5] (hh5) {};
	
	\node[red] (icr) at (hh2) {$\mathlarger{\mathlarger{\mathlarger{\mathlarger{\mathlarger{\bm{\times}}}}}}$};
	\node[red] (icr) at (hh4) {$\mathlarger{\mathlarger{\mathlarger{\mathlarger{\mathlarger{\bm{\times}}}}}}$};
	
	\node[circle, draw, thick, right=4em of hh2] (o1) {};
	\node[circle, draw, thick, right=4em of hh4] (o2) {};
	
	\draw[-stealth, thick] (i3) -- (h2);
	\draw[-stealth, thick] (i3) -- (h5);
	\draw[-stealth, thick] (i4) -- (h2);
	\draw[-stealth, thick] (i4) -- (h5);
	\draw[-stealth, thick] (i5) -- (h2);
	\draw[-stealth, thick] (i5) -- (h5);
	
	\draw[-stealth, thick] (h2) -- (hh1);
	\draw[-stealth, thick] (h2) -- (hh3);
	\draw[-stealth, thick] (h2) -- (hh5);
	\draw[-stealth, thick] (h5) -- (hh1);
	\draw[-stealth, thick] (h5) -- (hh3);
	\draw[-stealth, thick] (h5) -- (hh5);
	
	\draw[-stealth, thick] (hh1) -- (o1);
	\draw[-stealth, thick] (hh1) -- (o2);
	\draw[-stealth, thick] (hh3) -- (o1);
	\draw[-stealth, thick] (hh3) -- (o2);
	\draw[-stealth, thick] (hh5) -- (o1);
	\draw[-stealth, thick] (hh5) -- (o2);

\end{tikzpicture}
\end{document}

    %\end{Large}
    \caption{Illustrating dropout functionality}
    \label{fig:Dropout_function}
\end{figure}

\subsubsection*{Dropout}
DNNs contain multiple non-linear hidden layers and which makes them easily learn
complex relationships between their inputs and outputs. With a small training set, this
relationship adds sampling noise that won't exist in the real-world data even if drawn
from the same distribution. This leads to overfitting and several methods have been
developed to reduce its effect.
\begin{enumerate}
    \item early stopping as soon as the validation error gets worse than the training
        error.
    \item L1 and L2 regularisation which penalises the weights \cite{Schmidhuber_2015}.
    \item Randomly drop units(along with their connection) from the neutral network during
        training \cite{dropoutpaper}. Figure \ref{fig:Dropout_function} illustrates how to
        do the random dropping of units.
    \end{enumerate}

\section{Deep deep learning}



\subsection{CNN}
\subsubsection*{convolution - kernels, strides, padding}
\subsubsection*{max pooling}
\subsubsection*{batch normalisation}
\subsubsection*{flatten}
\subsubsection*{fully connected}
\subsection{RNN}
\subsubsection{LSTM}
\section{Sensors}
\subsection{visual sensors}
\subsubsection*{RGB}
\subsubsection*{depth}
\subsubsection*{segmentation}
\subsection{measurement sensors}
\subsubsection*{radar}
\subsubsection*{control}

\section{data fusion}
\subsection{types of data fusion}
%\subsection{data fusion techniques}
\section{keras}
\subsection{functional api}
\subsubsection*{different layers}
\subsection{callbacks}
\subsubsection*{model checkpoint}
\subsubsection*{early stopping}
\subsubsection*{tensorboard}

\section{ROS}
\subsection{ROS2}
\subsubsection*{Sub,pub,msg filter, cache, slop, callbacks, spin, topic, bridge, message
types}







\begin{figure}[h]
	\centering
        \def\svgwidth{0.8\textwidth}
\begin{tiny}
        \input{figures/inkscape/simplernn.pdf_tex}

\end{tiny}

    \caption{simple rnn}
    \label{fig:rnn}
\end{figure}


\begin{figure}[h]
	\begin{center}
	   \def\svgwidth{0.8\columnwidth}
%    \includestandalone[width=\textwidth]{figures/fig/lstmtikz}
    \input{figures/inkscape/lstm.pdf_tex} %use full path to know the location of pdftex
	\end{center}
    \caption{lstm}
    \label{fig:lstm}
\end{figure}


\iffalse

\begin{figure}
	\centering
    \includestandalone[width=\textwidth]{figures/fig/SLsetup}
    \caption{Supervised Learning set up}
    \label{fig:SL_setup}
\end{figure}

\begin{figure}
	\centering
    \includestandalone[width=0.5\textwidth]{figures/fig/2d_convolution}
    \caption{Two dimensional convolution}
    \label{fig:2dconv}
\end{figure}


\begin{figure}
	\centering
        \def\svgwidth{0.8\textwidth}

        \input{figures/inkscape/lossfunction.pdf_tex}

    \caption{Loss function}
    \label{fig:loss function}
\end{figure}

\begin{figure}
	\centering
        \def\svgwidth{0.81\textwidth}

        \input{figures/inkscape/optimizer.pdf_tex}

    \caption{With optimizer}
    \label{fig:withoptimizer}
\end{figure}

\fi


