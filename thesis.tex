%\documentclass[11pt,twoside,openany]{is2}
\documentclass[11pt,twoside,openright]{styles/is2}
%%
%% TOGGLE TO MAKE REMARKS VISIBLE
%%
%%\newcommand{\remark}[1]{\begin{quotation}\textit{#1}\end{quotation}}
%%\newcommand{\remark}[1]{}

\usepackage[hyperfootnotes=false, colorlinks=false, pdfborder={0 0 0}]{hyperref}
\usepackage{psfrag}
\usepackage{graphics}
\usepackage{graphicx}
%\usepackage[T1]{fontenc}
%\usepackage[latin1]{inputenc}
\usepackage[english]{babel}

\usepackage{boxedminipage}
\setlength{\fboxsep}{4mm}
\usepackage{todonotes}
\usepackage[inline]{enumitem}
\usepackage{float}
%\usepackage{afterpage}
%\usepackage{subfig}
%\usepackage{theorem}
%\usepackage{tabularx}
%\usepackage{helvet}
%\usepackage{a4wide}
\usepackage{times}
%\usepackage{palatino}
\usepackage{makeidx}
\usepackage{amsmath}
\usepackage{amssymb}
%\usepackage{mathptm}
%\usepackage{moreverb}
%\usepackage{picinpar}
%\usepackage{fancyhdr}
%\usepackage{alltt}
%\usepackage{fancybox}
%\usepackage{eepic}
\usepackage{wrapfig}
\usepackage{epsfig}
%\usepackage{doxygen}
\usepackage{styles/hangcaption}
%\usepackage[dvips]{rotating}
\usepackage{rotating}
\usepackage[hang]{subfigure}
\usepackage{styles/algo}
%\usepackage{textcomp,xspace}
\usepackage{listings}
%For inkscape \usepackage{import}
\usepackage{xifthen}
\usepackage{pdfpages}
%Need for inkscape coloured and transparent images
\usepackage{transparent}
\usepackage{xcolor}
\usepackage{calc}
\graphicspath{{figures/inkscape/}} %Location of pdf files.
%Export object's size than
%documents. %for Tikz standalone package
\usepackage[subpreambles=true]{standalone}
\usepackage{tikz}
\usetikzlibrary{shapes,external}
\usepackage{svg}
\usetikzlibrary{arrows,backgrounds}
\usepgflibrary{shapes.multipart}
%\renewcommand{\sfdefault}{cmss}
%\usepackage{subcaption}
\renewcommand{\floatpagefraction}{0.75}
\sloppy
\setlength{\parskip}{4pt}
\setlength{\parindent}{0pt}
\usepackage{listings}
%\pagestyle{headings}
%\makeindex

\setcounter{tocdepth}{2}
\hyphenation{ana-ly-sis opera-tion opera-tions opera-tio-nal neglected}


\newcommand{\rot}[1]{

\begin{rotate}{-60}

#1

\end{rotate}

}


\newcommand{\upw}[1]{

\begin{rotate}{90}

#1

\end{rotate}

}

\newcommand{\longpage}{\enlargethispage{\baselineskip}}
\newcommand{\shortpage}{\enlargethispage{-\baselineskip}}



% macht durch Kapitelwechsel entstandene leere Seiten auch wirklich leer

% entnommen aus der Doku zum fancyhdr Paket Ver. 1.99d

\makeatletter

\def\cleardoublepage{\clearpage\if@twoside \ifodd\c@page\else

\hbox{}

\vspace*{\fill}

\begin{center}

%This page was intentionally left blank.

\end{center}

\vspace{\fill}

\thispagestyle{empty}

\newpage

\if@twocolumn\hbox{}\newpage\fi\fi\fi}

\makeatother


\definecolor{codegreen}{rgb}{0,0.6,0}
\definecolor{codegray}{rgb}{0.5,0.5,0.5}
\definecolor{codepurple}{rgb}{0.58,0,0.82}
\definecolor{backcolour}{rgb}{0.95,0.95,0.92}

\lstdefinestyle{mystyle}{
    backgroundcolor=\color{backcolour},
    commentstyle=\color{codegreen},
    keywordstyle=\color{magenta},
    numberstyle=\tiny\color{codegray},
    stringstyle=\color{codepurple},
    basicstyle=\ttfamily\footnotesize,
    breakatwhitespace=false,
    breaklines=true,
    captionpos=b,
    keepspaces=true,
    numbers=left,
    numbersep=5pt,
    showspaces=false,
    showstringspaces=false,
    showtabs=false,
    tabsize=2
}

\lstset{style=mystyle}
%%%%%%%%%%%%%%%%%%%%%%%%%%%%%%%%%%%%%%%%%%%%%%%%%%%%%%%%%%%%%%%%%%%%%%%%%%%%%%%
%                        BEGIN OF THE DOCUMENT                                %
%%%%%%%%%%%%%%%%%%%%%%%%%%%%%%%%%%%%%%%%%%%%%%%%%%%%%%%%%%%%%%%%%%%%%%%%%%%%%%%


\begin{document}
% bstctlcite MUST be at the very beginning of your document,
% otherwise, it will "miss" some/most of your cites
\bstctlcite{IEEEtranBST:BSTcontrol}

%\title{Deep Learning for Autonomous Driving using End-to-End learning and Data Fusion}
%\title{Exploration and Implementation of LEON3\\ SPARC V8 Architecture Using LISA}
\authortitel{cand.-ing.}
\author{Deepak Ramani}
\signature{Deepak Ramani}
\typ{Master Thesis M107}
\betreuer{Shawan Taha Mohammed}
\matnr{314726}
\date{Sep 30, 2020}

\monat{September, 2020}
\maketitleis2

%%% Local Variables: 
%%% mode: latex
%%% TeX-master: "diplomarbeit"
%%% End: 

\tableofcontents

\newlength\dunder
\settowidth{\dunder}{\_}
\newcommand{\twound}{\rule{2\dunder}{0.4pt}}
\newcommand{\oneund}{\rule{\dunder}{0.4pt}}
\chapter{Introduction}

The last decade has seen massive growth in the field of Autonomous
Driving, primarily due to proliferation of graphical processing unit(GPU), and several projects like Google(Waymo) \cite{Waymo},
Berkeley-DeepDrive \cite{Berkeley-DeepDrive}, Apollo \cite{Apollo}, making their datasets
open-source which have made it easier for people to work on these data and achieve better performance gains.

Training a deep neural network(DNN) forms the core of making a car autonomous.
By using supervised learning, one can achieve reliable results as it gives greater control
at each stage of training. The data-driven approach collects data in advance and labels it
appropriately. It can then be fed to the DNN using supervised
learning algorithms to train the best model possible.

Ever since the discovery of Alexnet in 2012 \cite{Alexnet2012}, the convolutional neural network(CNN) and
deep learning(DL) are preferred choices to analyse images.  However, it is well known that the camera sensors are susceptible even to a slight change in weather conditions.
Sensors like radar \cite{Radar}, LIDAR \cite{LIDAR}, ultrasonic\cite{ultrasonic}, depth camera
give additional depth information for obstacle detection. These features are then fused
with the camera images and this process is called data fusion.

Even though there are some public data available, it is still not enough to reliably
train a DNN. Then there is  the cost of building an autonomous car. Fortunately, the last
years have seen growth in reliable simulators which
helped massively to collect data to help explore this field of research.
To name a few simulators that are being actively used -- LGSVL \cite{rong2020lgsvl}, Nvidia Drive
\cite{NvidiaSimulator}, Carla \cite{CarlaSimulator}, CarMaker \cite{CarMaker}.
In this thesis, the LGSVL simulator is used.

\begin{figure}[h]
    \begin{center}
        \includegraphics[width=\textwidth]{figures/png/intro/scrot_lgsvl_2.png}
    \end{center}
    \caption{LGSVL\cite{rong2020lgsvl} simulator active with all sensors}
        \label{fig:LGSVL_constellation_sensors}
\end{figure}

The LGSVL simulator allows the use of different sensors with minimal effort. The data
from different sensors are published through websocket. So to capture these data, we
need an interface/protocol which can understand the sent data's message type and enable the
receiving node to store them. However, the data from each sensor arrives at
different rates. Hence it is necessary to collect and synchronise them in the order of their arrival
before storing, so as to not lose their integrity and thereby prevent corrupting the dataset.
Robotic operating system(ROS) \cite{ROS2} and its functionalities fulfil
this purpose. It allows seamless transfer of simulator's data by subscribing to sensor
nodes in the form of topics. Then the subscribing node with the help of ROS libraries, synchronises it as necessary for storage.

So, the data that resembles real-world is stored locally for later analysis and
research.


\section{Motivation}

The motivation for this thesis is to use a simulator, do the required tests, and
determine whether using a simulator does indeed help in perceiving the environment
and accomplishing the goal of driving in the real-world.

One of the major obstacles in autonomous driving is the cost associated with integrating
sensors in addition to manufacturing a vehicle. Representing the environment around the vehicle(ego vehicle) requires information from all in-car sensors.
The resources demanded to make an optimal decision are also a challenge.

The high cost of associated sensors such as LIDAR\cite{mesarticleonLidar}, has put off many smaller research groups from using them
in their work. Simulation allows conducting adequate tests(at a low cost) and quicker
development of algorithms. Simulator provides a safer environment to test and debug these
algorithms.

So with the help of a simulator it is observed how different constellation of sensors work, how different modalities interact with
each other, and what impact these factors have on the overall performance of the deep
neural networks.

Finally, an end-to-end system is implemented which simulates real-world behaviour and gives
results which can then be applied to future research to make it more robust.
%%%%%%%%%%%%%%%
%TODO -
%insert a table 2.1 in page10 showing how different sensor combo work. Use the thesis under
%reference in firefox.
%https://www.researchgate.net/profile/Markus_Weber14/publication/342283221_Autonomous_Driving_Radar_Sensor_Noise_Filtering_and_Multimodal_Sensor_Fusion_for_Object_Detection_with_Artificial_Neural_Networks/links/5eebe41d458515814a6aa417/Autonomous-Driving-Radar-Sensor-Noise-Filtering-and-Multimodal-Sensor-Fusion-for-Object-Detection-with-Artificial-Neural-Networks.pdf
%%%%%%%%%%%%%%%%

\section{Goal}
    The desired goals of this thesis are listed below:

    \begin{enumerate}
    \item Building a basic autonomous driving framework which comprises of the following three
        components:
        \begin{itemize}
        \item ROS - use ROS2 to synchronise the data received from the simulator and its plugin through a
            rosbridge, use functionalities such as slop and cache, to sort the data according to
            their received time in order not to scramble the information. During the evaluation,
            use the same functionalities to send command controls back to the simulator.
        \item Rosbridge - use a bridge transport protocol that connects the ros to the simulator.
        \item Docker - set up a work environment that is independent of hardware or
            operating system which allows easy running of the commands for data collection
            and evaluation.
        \end{itemize}
    \item Implement a system that can efficiently collect and label data.
    \item Implement an end-to-end neural network architecture which applies state of the
        art deep learning techniques to learn driving by predicting the steering and other
        control commands from image pixels.
    \item Implement and analyse different constellation of sensors with different data
        fusion techniques.
    \end{enumerate}

\section{Related Work}
In 2012, Alexnet \cite{Alexnet2012} used CNNs to do object classification which, then
in Computer Vision became the dominated approach for classification. Both Chen \textit{et
al.} \cite{chen2017} and Bojarski \textit{et al.} \cite{bojarski2016end} extended
\cite{Alexnet2012}'s approach of using CNN and showed that in addition to classification, CNN can
extract features from images. Then they went on to demonstrate through an end-to-end
network(which self-optimises itself based on its inputs), that steering angles can be
predicted to keep the car in the lane of a road.

In a different field, but using CNN, Sergey Levine \textit{et al.}
\cite{GooglePaperonCNNActuation} in 2016 corroborated that it was indeed possible to extract
features with CNN and predict motor control actions in \textit{object picking robots}.

Then, Xu \textit{et al.} \cite{XuGYD16CNNLSTM} in the same year with CNN-LSTM architecture
showed that using the previous ego-motion events helped predict future ego-motion events.
Using CNNs in an end-to-end architecture raised some questions on how it reached its
decisions. So in 2017, both \cite{heatmapsLearning}, \cite{BojarskiCNN1} did visual
analysis after the CNN layers to better understand the module's functionality.
Vehicle control is more than just steering control. For smoother control, acceleration and
braking are necessary besides steering. Both acceleration and deacceleration are dependent on  the user's driving
style, lane speed limit and traffic etc. Yand \textit{et
al.} \cite{E2EMultimodalDiscreteSpeed} used CNN-LSTM architecture and provided the LSTM
with feedback speed to determine the velocity of the ego vehicle.

Besides vehicle control, perceiving the environment is necessary for collision avoidance.
The RGB colour camera sensors don't provide the depth information which is critical for collision avoidance.
Hence, it is essential to fuse other sensors with diverse modalities with RGB to predict an optimal output.
Liu \textit{et al.} \cite{liu2018learn} provided rules in fusing data. They said that it was
essential to pick out only vital information and discard other noisy data.
They also described the techniques involved in data fusion -- early/late
fusion, hybrid fusion, model ensemble and joint training. Park \textit{et
al.} \cite{ParkHBB16} gave us methods to enhance the features by using feature amplification
or multiplicative fusion. Zhou \textit{et al.} \cite{ZhouSideChannel} detailed how fusing
data into CNN affects the overall performance.

Even though the fused dataset gives a performance boost, it performs worse
compared to individual modality. The combined fused model overfits more
than its counterparts. The fundamental drawback of
\textit{gradient descent} in backpropagation causes the networks to overfit. This paper \cite{wang2020makes} introduced a technique
called \textit{gradient blending} to counteract this problem.

Xiao \textit{et al.}\cite{XiaoCodevillaMultimodalE2E} applied all the fusion techniques
mentioned above with an imitation based end-to-end network\cite{codevilla2017endtoend}.
They concluded that RGB images with depth information(obtained through a different modality)
could indeed result in better performing end-to-end network model.

\section{Contribution}
For supervised learning task, it is essential to collect data and label them. LGSVL, ROS provide an excellent platform to carry out this work. 

So with their help, it was possible to acheive 
\begin{enumerate}
    \item For first time implementation of a basic autonomous driving framework based on ROS and LGSVL.
    \item An implementation of an end-to-end training neural network with LGSVL for the first time.
    \item Fusing of data with different sensor in different constellations with LGSVL. 
\end{enumerate}
  



\chapter{Fundamentals}
\section{Machine learning: What and why?}
Machine learning is all about of learning from data; gaining knowledge from it. More the
data, more the chances to learn. Machine learning was initially thought of as automating
redundant human tasks and later developed into something that allowed solving complex
mathematical problems. It was seen as an addition to humans than extension of them.
Machine learning these days are required to perform tasks that are quite obvious and
natural to humans; such as recognising faces in images or perceive the road, environment
around the vehicle and make decisions instinctively.

All these attributes require to extend the field of machine learning to extract more
knowledge from a given data. The figure \ref{fig:ai_ml_dl} shows how artificial
intelligence(AI) is divided to specific areas -- Machine learning(ML) and deep learning(DL).

So, in this chapter, an overview is given on the concepts that are used in the later chapters.

\begin{figure}[h]
	\centering
    \def\svgwidth{0.5\textwidth}
    \input{figures/inkscape/aimldl.pdf_tex} %use full path to know the location of pdftex
    \caption{Schema of AI, ML and DL}
    \label{fig:ai_ml_dl}
\end{figure}

\subsection{Learning algorithms}
Machine learning provides a means to tackle tasks that are complex to solve through fixed
programmes and designed by human beings \cite{Goodfellow-et-al-2016}. A learning algorithm
is an algorithm which gains the ability to learn from data. A ML algorithm is one that
gains the ability to learn from an experience E with respect to some class of tasks T and
performance measure P \cite{mitchell1996m}.

\subsubsection*{Tasks, T}
The two major tasks in ML are \textit{classification} and \textit{regression}.

In classification related tasks, the system is identify which of \textit{k} categories an
input belongs to. A function $f : \mathbb{R}^n \rightarrow\{1, \ldots,k\}$ is used by the
learning algorithm to solve this task. When $y = f(x)$, the model assigns an input
described by vector $\mathbf{x}$ to a category
identified by numeric code $y$. There are other variants of the classification
task, for example, where $f$ outputs a probability distribution over classes
\cite{Goodfellow-et-al-2016_1}. Alexnet \cite{Alexnet2012} is one of the examples of
classification tasks that performed object recognition.

Regression is similar to classification except that the output is a continuous value. A function $f
": \mathbb{R}^n \rightarrow \mathbb{R}$ predicts a numerical value for some input.
Predicting the steering control value is a prime example for regression task.

There are ofcourse other tasks but only classification and regession are used in this
thesis. Hence the narrow focus.

\subsubsection*{Performance measure, P}
To evaluate the performance of a ML algorithm, it is must to design quantitative measure
of its performance. Usually this performance measure P is specific to the task T. There
are two distinct types of measurements -- accuracy and error rate.

If the goal is to learn a mapping from inputs $x$ to outputs $y$, where $y \in \{1,\ldots
, C\}$, with $C$ being the number of classes. If $C = 2$, this is
called binary classification (in which case we often assume $y \in \{0, 1\})$; if $C > 2$, this is called
multiclass classification. If the class labels are not mutually exclusive (e.g., somebody may be
classified as tall and strong), we call it multi-label classification
\cite{murphy2013machine_1}.

Accuracy is a proportion of how much the model produces the correct outputs.
So in the case of binary classification, if the function $f$ predicts a probability
densities $\hat y \in \{0.3, 0.7\}$, for a ground truth $y$ of value $1$, then P is $70\%$
accurate or the error rate is $30\%$.

It is must that the model is evaluated with a data that it has not seen before. This data
\textit{testing set}, gives a good judgement on the performance.

\subsubsection*{Experience, E}
The ML algorithms can be classified into \textit{supervised}, \textit{unsupervised} and
\textit{reinforcement} learning based on the kind of experience they are allowed to have.
A learning algorithm is allowed to gain experience by going through the \textit{dataset}.
A dataset is collection of all the examples for a given task. For example, to classify
which category a shown image belongs to has collection of images as dataset
\cite{cifar10}. Sometimes datasets are also called as \textit{data points}.

The focus will be on supervised learning in our case. The CIFAR
dataset \cite{cifar10} containing images as features has \textit{targets} or
\textit{labels} associated with it. In supervised learning(SL), the target functionality
is shown to the learning algorithm. A random vector $\mathbf{x}$ explicitly attempts to
learn the probability distribution $p(\mathbf{x})$ and predicts $\mathbf{y}$ from
$\mathbf{x}$, usually estimating $p(\mathbf{y}\mid\mathbf{x})$.

\section{Deep Learning}
Deep learning is a subset of machine learning. It takes all the algorithms, concepts from
the machine learning and narrows the focus to enable a model learn from data to do tasks
that involve less human interaction, large number of data and parameters.

\subsection{Simple neural network}
\textit{Linear regression} is one of the common SL algorithms. It solves the regression
problem. For example, if there is vector $\mathbf{x} \in \mathbb{R}^n$ as input and
predict a scalar value $y \in \mathbb{R}$ as its output, then in linear regression, output
is a linear function of the input. We can define it as
\begin{equation}
    \hat y = \mathbf{w}^T\mathbf{x}
\end{equation}
where $\mathbf{w} \in \mathbb{R}^n$ is a vector of parameters.

$\mathbf{w}$ is usually referred to as a set of weights that determine how each feature
affects the prediction. A $\mathbf{w}_i$ is simply multiplied with a feature $x_i$ to
predict $\hat y$. By manipulating the $\mathbf{w}_i$ value, the corresponding feature has
an effect on the prediction  $\hat y$.

A learning algorithm, in this case linear regression, is implemented as a perceptron. It
is a single-layer neural network. They generally consists of four main parts -- input
nodes $x_i$, weights $w_i$, bias $b_0$(if necessary), net sum $\Sigma$ and an activation
function $\sigma$. This is shown in the figure \ref{fig:simpleNN}.

\begin{figure}[h]
    \centering
        \def\svgwidth{0.5\textwidth}
        \input{figures/inkscape/multilayer_perceptron.pdf_tex}
        \caption{A simple neutral network}
        \label{fig:simpleNN}
\end{figure}

\subsubsection*{Activation function}
The common activation functions used are Rectified Linear unit(ReLu), Sigmoid, tanh and
softmax function. For each type of activation, $\sigma$ then decides if the input received is
relevant or not relevant. To convert linear inputs to non-linear, all that has to be done
is to use a non-linear activation function. In the figure \ref{fig:activationfunctions},
shows the characteristics of some of the activation functions.
\begin{figure}[ht]
	\begin{center}
   \def\svgwidth{0.5\textwidth}
    \input{figures/inkscape/activationfn.pdf_tex} %use full path to know the location of pdftex
	\end{center}
    \caption{Activation functions}
    \label{fig:activationfunctions}
\end{figure}
For classification tasks, usually the last layer of the networks is equipped with softmax
activation layer. This function normalises the output to a probability distribution over
predicted output classes.

\subsubsection*{Multilayer feedforward networks}
Deep feedforward networks or multilayer perceptrons are the quintessential deep learning
models. Its goal is to approximate function $f^*$. In the below figure \ref{fig:MLP},
information flows from inputs $\mathbf{x}$ to output $y$ using a mapping function
$\mathbf{y} = f(\mathbf{x};\mathbf{\theta})$ where $\theta$ are the parameters values
which the MLP learns for optimal approximation.

They are called feedforward as there are no feedback connections in which outputs of the
model are fed back into itself. Feedforwards networks with feedbacks are called
\textit{recurrent neutral networks}.

Feedforwards networks form the core for many commercial applications. For example, the
convolutional neural networks used for object detection are a special kind of feedforward
networks.
\begin{figure}[h]
    \def\svgwidth{0.5\textwidth}
	\begin{center}
        \input{figures/inkscape/inputhiddenoutput.pdf_tex} %use
    \end{center}
    \caption{Multi layer perceptrons}
    \label{fig:MLP}
\end{figure}

More the hidden layers, more the depth of the feedforward networks. The width is given by
the dimensionality of the hidden layer.

\subsubsection*{Loss function}
As mentioned before, a mapping function $f$ noisily approximates the input $x$ to output
$y$. So, the noise or the deviation from the true value(ground truth) must be kept at
minimum. The function that calculates the deviation is called \textit{cost} or
\textit{loss} function. It is important to choose the right loss function for a model.

\begin{figure}[h]
	\centering
    \def\svgwidth{0.5\textwidth}
    \input{figures/inkscape/mse.pdf_tex} %use full path to know the location of pdftex
    \caption{Mapping from x to y. The predictor is shown as linear line. The distance
    between the true values and predictor gives the loss. The sum of all the distances
gives the loss function.}
\label{fig:loss function}
\end{figure}

For multi-label classification tasks, \textit{categorical cross-entropy} function is used.
For each category, cross-entropy is calculated. The difference between the cross-entropy
of training data and the model's predictions is the cost function.

For regression tasks, the models are subjected to loss functions such as \textit{mean
absolute error}(MAE), \textit{mean squared error}(MSE) and \textit{mean squared
logarithmic error}(MSLE). In MAE, the mean of absolute differences among predictions and
expected results are calculated.
\begin{equation}
    MAE = \frac{1}{n}\sum_{i=1}^n\left |y_i -\hat y_i \right|
\end{equation}
In MSE, the mean of squared differences among predictions and true outputs are
calculated.
\begin{equation}
    MSE = \frac{1}{n}\sum_{i=1}^n (y_i - \hat y_i)^2
\end{equation}
In MSLE, the mean of relative distances between predictions and true outputs are
calculated.
\begin{equation}
    MSLE = \frac{1}{n}\sum_{i=1}^n(log(y_i+1)-log(\hat y_i+1))^2
\end{equation}

\subsubsection*{Gradient descent} \label{gradientdescent}
Gradient descent is another technique to minimise the cost function parameterised by a
model parameter $\mathbf{w}$. The first derivative(or gradient) gives the slope of the
cost function. Hence, to minimise it, direction opposite to the gradient is chosen.

\begin{figure}[h]
	\centering
    \def\svgwidth{0.5\textwidth}
        \input{figures/inkscape/minima.pdf_tex}
    \caption{Finding the stochastic gradient descent}
    \label{fig:gradientdescent}
\end{figure}

The rate at the which the gradient step reduces is given by the \textit{learning rate}. If
the learning rate is high, greater the step size of each gradient; possibly causing the
step to miss the global minima. Lower the learning rate, more steps or training cycles
needed to reach the global minima. So the rate must be carefully chosen for each model.

\subsubsection*{Backpropagation}
Backpropagation are a class of algorithms which help in training feedforward neural
networks for supervised learning. A model is said to fit when the gradient computation of
the loss function is efficient w.r.t the weights of single input-output in the network.
Backpropagation performs the effective gradient computation using the loss functions
explained above.

\subsubsection*{Optimizer}
The loss function was able to explain how far the predictions were compared to the true
outputs in a mathematical way. During training process, certain parameters can be tweaked
to help the loss function predict correct and optimised results. However, there are
question such as how to change them, by how much and when?

This is exactly optmizer's function. As explained in \ref{gradientdescent }, gradient
descent and learning rate form the core of optimizer's functionality. \textit{Stocastic
gradient descent}(SGD) is one of the oldest techniques in which gradients for all of your
training examples on every pass of gradient descent are calculated. However, they are slow
and require much computation power. Some of the other popular optimizers are Adam,
Adaguard, RMSprop. In this work, Adam is used. Adam stands for adaptive moment estimation.
It is a combination of all the advantages of two other extensions of SGD -- Adaguard and
RMSprop. Adam is computationally efficient, straight forward to implement, invariant to
diagonal rescale of the gradients, less effort need to hyperparameters tuning.

\subsubsection*{Challenges in Machine learning algorithms}
\begin{enumerate}
    \item insufficient labelled data
    \item poor quality data and irrelevant features
    \item overfitting/underfitting the model
\end{enumerate}

The first two issues can be solved if the user is careful during data collection and does
preprocessing before feeding the data to the training model. However, if the training or
the test data is too small, the model is subject to underfitting or overfitting. Though
our aim is to reduce the error in the training set, we also need to reduce the error in
the test set. The gap between training and testing error is also important parameter.
Underfitting occurs when the model is not able to obtain sufficiently low error value for
the training set. And if the gap between training and testing error is too large,
overfitting happens. The sweet spot is to stop training the model when the gap between the
two sets is at a minimum value. Left of the optimal point, the model underfits. Right of
it, the model overfits. The below figure \ref{fig:overfittingunderfitting} shows it very
well. Validation error is the error calculated for the test set.
\begin{figure}[h]
	\centering
    \def\svgwidth{0.75\textwidth}
    \input{figures/inkscape/overfitting.pdf_tex} %use full path to know the location of pdftex
    \caption{Relationship between capacity and error. Inspired from
    \cite{Goodfellow-et-al-2016}}
    \label{fig:overfittingunderfitting}
\end{figure}


\begin{figure}[h]
	\centering
    \def\svgwidth{0.5\textwidth}
   % \begin{Large}
    \input{figures/inkscape/dropout.pdf_tex}
    %\end{Large}
    \caption{Illustrating dropout functionality}
    \label{fig:Dropout_function}
\end{figure}

\subsubsection*{Dropout}
DNNs contain multiple non-linear hidden layers and which makes them easily learn
complex relationships between their inputs and outputs. With a small training set, this
relationship adds sampling noise that won't exist in the real-world data even if drawn
from the same distribution. This leads to overfitting and several methods have been
developed to reduce its effect.
\begin{enumerate}
    \item early stopping as soon as the validation error gets worse than the training
        error.
    \item L1 and L2 regularisation which penalises the weights \cite{Schmidhuber_2015}.
    \item Randomly drop units(along with their connection) from the neutral network during
        training \cite{dropoutpaper}. Figure \ref{fig:Dropout_function} illustrates how to
        do the random dropping of units.
    \end{enumerate}

\section{Deep deep learning}



\subsection{CNN}
\subsubsection*{convolution - kernels, strides, padding}
\subsubsection*{max pooling}
\subsubsection*{batch normalisation}
\subsubsection*{flatten}
\subsubsection*{fully connected}
\subsection{RNN}
\subsubsection{LSTM}
\section{Sensors}
\subsection{visual sensors}
\subsubsection*{RGB}
\subsubsection*{depth}
\subsubsection*{segmentation}
\subsection{measurement sensors}
\subsubsection*{radar}
\subsubsection*{control}

\section{data fusion}
\subsection{types of data fusion}
%\subsection{data fusion techniques}
\section{keras}
\subsection{functional api}
\subsubsection*{different layers}
\subsection{callbacks}
\subsubsection*{model checkpoint}
\subsubsection*{early stopping}
\subsubsection*{tensorboard}

\section{ROS}
\subsection{ROS2}
\subsubsection*{Sub,pub,msg filter, cache, slop, callbacks, spin, topic, bridge, message
types}







\begin{figure}[h]
	\centering
        \def\svgwidth{0.8\textwidth}
\begin{tiny}
        \input{figures/inkscape/simplernn.pdf_tex}

\end{tiny}

    \caption{simple rnn}
    \label{fig:rnn}
\end{figure}


\begin{figure}[h]
	\begin{center}
	   \def\svgwidth{0.8\columnwidth}
%    \includestandalone[width=\textwidth]{figures/fig/lstmtikz}
    \input{figures/inkscape/lstm.pdf_tex} %use full path to know the location of pdftex
	\end{center}
    \caption{lstm}
    \label{fig:lstm}
\end{figure}


\iffalse

\begin{figure}
	\centering
    \includestandalone[width=\textwidth]{figures/fig/SLsetup}
    \caption{Supervised Learning set up}
    \label{fig:SL_setup}
\end{figure}

\begin{figure}
	\centering
    \includestandalone[width=0.5\textwidth]{figures/fig/2d_convolution}
    \caption{Two dimensional convolution}
    \label{fig:2dconv}
\end{figure}


\begin{figure}
	\centering
        \def\svgwidth{0.8\textwidth}

        \input{figures/inkscape/lossfunction.pdf_tex}

    \caption{Loss function}
    \label{fig:loss function}
\end{figure}

\begin{figure}
	\centering
        \def\svgwidth{0.81\textwidth}

        \input{figures/inkscape/optimizer.pdf_tex}

    \caption{With optimizer}
    \label{fig:withoptimizer}
\end{figure}

\fi



\chapter{Simulation and Simulator}
From the beginning of autonomous driving research, simulators have played a key role in
development and testing new algorithms. Simulators allow developers to quickly test their
algorithms without driving real vehicles. In this chapter, we will the conditions a
simulator must satisfy, and go in detail about LGSVL simulator and its development.
\section{Need for a simulator}
One of the important questions to ask before explaining about simulation is to understand
why should one need a simulator to do simulation. As explained in the previous chapter
(\ref{chapter:fundamentals}), deep neural network(DNN) using supervised learning algorithm
needs huge amount of labelled data. Since the cost of collecting that amount of data in
real road vehicle is too expensive, researchers have sought the help of simulators. A
simulator is an application which simulates a real-world environment, virtually. With the
help of a simulator, one can collect any amount of data they wish for their project.

\section{Conditions for a simulator}

Data collection is one of the most important phases in supervised learning. So caution must
be taken in choosing a simulator. A simulator must fulfil certain conditions to be
qualified as a good one.
\begin{itemize}
    \item It must have a vehicle that can move around in a virtual map.
    \item The vehicle must be equipped with appropriate sensors for perceiving the
        environment properly.
    \item The virtual map must try to mirror the real-world to an extent. That mean it
        should have proper terrain to drive around, lane marking for lane detection, other
        cars to mirror the real world traffic, pedestrians, and real world weather
        conditions.
    \item It must provide a medium to collect data and allow interfaces to transfer the
        data. It should also be able to receive data in case the user needs to validate
        the data collected.
    \item Finally and most importantly, support end-to-end, full stack simulation.
\end{itemize}

\section{LG SVL simulator}
A simulator chosen for this thesis is from LG research centre in Silicon Valley,
California called LGSVL simulator. It is an open source project where the code is
regularly published at Github \cite{lgsvlgithub}. This simulator satisfies all the
conditions listed above. They provide an out-of-the-box solution which can meet the
needs of developers wishing to focus on testing their autonomous vehicle algorithms. It
also supports Apollo \cite{ApolloAuto} and Autoware \cite{autowarePaper}.

\subsection{LGSVL simulator development}
LGSVL simulator's core simulation engine is developed using the Unity game engine
\cite{unitygameengine}. Unity game engine is written in C\# programming language. Since a
game engine inherently supports animation, the simulator is able to extend that
functionality easily. In addition to Unity, also supports several libraries necessary to
compute complex mathematical operations. With Unity's latest High Definition Render
Pipeline(HDRP), LGSVL is able to simulate photo-realistic virtual environments that match
the real world.

\subsection{Overview of LG SVL simulator}

\subsubsection*{User AD Stack}
It supports user autonomous driving(AD) stack. That means a user can develop, test and verify through simulation.
The user AD stack connects to LGSVL Simulator through a communication bridge interface; a bridge is selected based
on the user AD stack’s runtime framework. This bridge interface can use a standard
protocol such ROS, ROS2 or custom one like CyberRT \cite{ApolloAuto}.

In addition LGSVL supports plug-in component which a user can develop and attach it to the
simulator. The simulator during runtime picks up this plug-in.

\subsubsection*{Simulation Engine}
As mentioned above, LGSVL uses Unity's latest HDRP game engine.

\subsubsection*{Sensor and vehicle models}
It supports sensor arrangement and importantly they are customisable. The sensors are
added and removed through JSON formatted text along with its parameters. These parameters
include sensor type, position of the sensor, topic name, publishing rate, and in some
sensors reference frame of measurement. Some of the popular sensors like camera sensors,
radar and LIDAR are supported. In addition, users can add their own custom sensors as
plug-in. Fig.\ref{fig:sensortypeslgsvl} gives a good overview of some of the sensors in action.

Vehicles provide a medium to travel the environment. Hence, vehicle dynamics is also
important.

\begin{figure}[!ht]
    \centering
    \def\svgwidth{0.9\columnwidth}
    \input{figures/inkscape/sensordistribution.pdf_tex}
    \caption{Different types of sensors in LGSVL simulator. Anticlockwise(from top): RGB
    colour camera, Segmentation camera, Radar(also 3D bounding boxes), LiDAR, Depth camera}
    \label{fig:sensortypeslgsvl}
\end{figure}

\begin{figure}[!ht]
    \centering
    \def\svgwidth{0.9\columnwidth}
    \input{figures/inkscape/lgsvlsensor1.pdf_tex}
    \caption{Different types of sensors in LGSVL simulator. Anticlockwise(from top): Depth
    camera, LiDAR, Radar(also 3D bounding boxes), Segmentation camera}
    \label{fig:sensortypeslgsvlnew}
\end{figure}

\subsubsection*{Environment and maps}
An environment, in this case, virtual, is a primary component in autonomous driving
simulation to provide many input to AD system. An environment affects almost all the
functionalities in a AD system such as perception, prediction and tracking modules. It
also affects the vehicle dynamics which is the key factor in vehicle control mechanism.
Through changes in the HD map, the environment affects localization and planning modules.
Finally, weather conditions such as rain, fog, night driving naturally affect the
environment. So caution must be taken while design the environment.

LGSVL supports creating, editing and exporting HD maps of existing 3d virtual environment.
3D environment also defines the rules about how agents must behave such as stopping at traffic
lights, giving way to priority traffic, respect lane boundaries etc.

As of writing, LGSVL supports virtual Sanfranciso city HD map. They also support smaller
maps like Shalun and Cubetown.

\begin{figure}[h]
    \centering
    \def\svgwidth{\textwidth}
    \input{figures/inkscape/weather.pdf_tex}
    \caption{LGSVL simulator in different weather conditions}
    \label{fig:weatherconditions1}
\end{figure}

\begin{figure}[h]
    \centering
    \def\svgwidth{\textwidth}
    \input{figures/inkscape/lgsvlweatherconditions.pdf_tex}
    \caption{LGSVL simulator in different weather conditions}
    \label{fig:weatherconditions}
\end{figure}

\subsubsection*{Test scenarios}
Test scenarios enable users to test their AD stack by simulating in an environment and
comparing and contrasting correct and expected behaviours. A lot of variables like HD maps,
traffic movement behaviour and their density, time of the day, weather conditions etc. also
play a role while testing. It is also possible to write scripts with the help of Python
API where scenarios can be created and tested.

Thus LGSVL simulator \cite{rong2020lgsvl} provides the best virtual environment to conduct our experiments for
autonomous driving.

\chapter{Implementation}
\label{chapter:implementation}
This chapter will present the implementation of end-to-end network with its extensions.
First we start with docker to set the environment. Then move on to LGSVL and ROS.
From there a closed loop is achieved to collect data, preprocess, introduce neural
network, implement the models, and evaluate it. After achieving the basic results for the
preliminary architecture, sensor fusion techniques are implemented.

\section{Docker}
Docker is an open-source platform for developing, shipping and running applications.
Because docker makes installing applications hardware independent, we use docker for our
tasks.
\begin{wrapfigure}{l}{0.5\textwidth}
	\centering
    \def\svgwidth{0.5\textwidth}
    \input{figures/inkscape/scrot_dockerengine.pdf_tex} %use full path to know the location of pdftex
    \caption{Docker Engine and its functions}
    \label{fig:dockerengine}
\end{wrapfigure}


\begin{figure}[t]
	\centering
    \def\svgwidth{\textwidth}
    \input{figures/inkscape/docker_1.pdf_tex} %use full path to know the location of pdftex
    \caption{Docker and its various functions}
    \label{fig:docker1}
\end{figure}

A docker architecture, as shown in \ref{fig:dockerarchitecure}, consists of client, host
and registry. To make all these components work, docker daemon is necessary. A daemon is a
type of long-running background process. The LGSVL docker image is pulled from the registry using
\textit{docker pull} command. An image is a read-only template with instructions for
creating a docker container. The instructions are provided using a \textit{dockerfile}.
One can then build the images by themselves or use an image that is already built. In our
case since the image is readily available, so we use it.

A docker container for each task can be defined. Along with a task, certain other
services may need to be run along with it. \textit{Docker compose} gives a perfect
solution to manage docker applications.  When setting the environment with docker is difficult in some cases, an anaconda
environment \cite{anacondaenv} is used.

\section{LGSVL simulator}
The LGSVL simulator is developed using Unity engine which is written in C\# language.
The LGSVL team organises their code base\cite{lgsvlgithub} in such a way that it makes it
easy for a beginner to learn the structure and implement new features or change the
existing ones.
\subsection{Web user interface and JSON sensor parameters}
The LGSVL team has developed a web UI to help users to configure maps, vehicles and
simulations. They also support different maps and vehicles configurations.
\begin{figure}
	\centering
    \def\svgwidth{0.8\textwidth}
    \input{figures/inkscape/LGSVLsoftwarearchi.pdf_tex} %use full path to know the location of pdftex
    \caption{LGSVL software architecture}
    \label{fig:lgsvlswarchitecture}
\end{figure}
A sensor configuration is defined in JSON format. If a user wishes to use a colour camera
sensor, then they need to use the JSON format appropriate for this sensor to the vehicle
configuration. Each sensor has a topic name which is then used by ROS nodes to subscribe
to it. In our case, we use a variety of sensors -- RGB colour camera, depth camera,
segmentation camera(output of a RGB camera fed to a neural network), and Radar sensor.
These sensors can be arranged/aligned in different constellations according to
requirements.

So, we have

\begin{enumerate}

    \item a RGB camera placed facing ahead parallel to the ground, another on
the left and right side of the car pointed an angle towards the ground,
    \item a depth camera following same configuration as RGB,
    \item a segmentation camera placed adjacent to the RGB front facing camera,
    \item a radar sensor placed front of the car near to the hood pointing ahead.
\end{enumerate}

The file associated with each sensor then picks up the values from JSON parameters and adjusts it in
run-time. If a sensor functionality is available, a user can use appropriate JSON to
utilise that sensor.

\subsection{Sensor plugin - Data collection and evaluation}
When a user doesn't want to disturb the current setup of the simulator but rather wants to add
some custom sensor to the vehicle configuration, then sensor plugin provides a perfect
solution. A set of guidelines must be followed while developing the plugin. In our case,
it is necessary to have a sensor plugin that would create a sensor and topics. This sensor
and these topics would then be used to fetch data from the simulator, transfer it through ROS bridge, and also receive
data for evaluation of the trained model. This custom plugin
extends the unity engine libraries to read the values from the JSON definition, fetch
values of steering, acceleration, braking from the vehicle control system, do
data transfer between simulator and ROS. The data transfer involves converting data types
to ROS understandable formats and convert back ROS to simulator formats during evaluation
phase.

\subsection{Radar Sensor}
Since data fusion is one of the goals of the thesis, using a radar sensor would provide
important depth information. However, in LGSVL, in its current version, the radar sensor is not working as
required. This necessitates some changes to some of the files in the LGSVL code base.
This process involves -
\begin{enumerate}
    \item Correcting the already existing radar sensor code to detect traffic properly and
        assign the data to their variables that look similar to ROS custom message standards.
    \item Converting the LGSVL data to ROS understandable  custom message formats.
    \item Adding ROS2 as the bridge type to establish between client and LGSVL.
    \item On the client side, editing the docker to include custom radar message types.
\end{enumerate}

LGSVL simulator is now configured to send data towards the client. In order to reach the
client, as mentioned before, a ROS bridge is needed. In the next section we will talk
about ROS and its uses.

\section{Data collection}
For data collection, \textit{python} programming language is used to create scripts. Each
script uses ROS components.

\begin{figure}
	\centering
    \def\svgwidth{\textwidth}
    \input{figures/inkscape/datacollection.pdf_tex} %use full path to know the location of pdftex
    \caption{Data collection module}
    \label{fig:datacollectionmodule}
\end{figure}

\subsection{ROS}
ROS, in our case, acts as an interface between simulator(server) and scripts(client).
We use ROS 2 and in particular \textit{dashing} iteration. The LGSVL follows the ROS standard for the message types.
The ROS nodes listen to the sensor topic(defined using JSON sensor parameter) and invoke
a callback whenever they receive data. Since each sensor receives at different rates, a
filter called message filters is used. With message filters, the queue size is set to a
higher value say 1000 and a delay(in seconds) through a \textit{slop} parameter of value
0.1 is used. This filter gathers all the subscribing nodes as one, synchronises
approximately to the delay parameter and invokes just one callback. This assures that data
from each listening node is present.

Inside the callback, the data is processed using computer vision(CV) and numpy libraries.
The ROS messages for the images include header and data
parts. The header part consists of the time at which the message is created and data part
contains the real data. With numpy libraries the real data is extracted easily for storage.

\subsection{ROS web bridge}
A ROS web bridge is a virtual bridge between scripts using ROS and LGSVL simulator. In this
case, a ROS 2 web bridge, written in nodejs, is established. It basically starts an instance
that listens to an IP address and its port. The LGSVL on the other side, listens to this
IP address and port. Hence a bridge is created to allow flow of data.
\begin{figure}[h]
	\centering
    \def\svgwidth{\textwidth}
    \input{figures/inkscape/ros2bridge.pdf_tex} %use full path to know the location of pdftex
    \caption{ROS2 web bridge implementation}
    \label{fig:ros2webbridge}
\end{figure}
\iffalse
\subsection{Building ROS2 package}
Before running the scripts with ROS, it must be built as ROS packages.  A package is a
container for ROS 2 code which makes it easier to share with others. Package creation in
ROS 2 uses \textit{ament} as its build system and \textit{colcon} as its build tool.
Packages can be created either in \textit{CMake} or \textit{Python}.

For CMake, \texttt{package.xml} and \texttt{CMakeLists.txt} files are necessary. \texttt{package.xml}
file contains meta information about the package. \texttt{CMakeLists.txt} file describes how to build the code within the package.

For Python, \texttt{setup.py}, \texttt{package.xml}, \texttt{setup.cfg} and
\texttt{resource/<package-name>} are needed. The \texttt{package.xml} file contains meta
information about the package. Unlike CMake, \texttt{setup.py} contains instructions on
how to install the package. The \texttt{setup.cfg} is required when a package has
executables, so \textit{ros2 run} can find them. Lastly there is
\texttt{resource/<package-name>}. It is a directory with the same name as the package,
used by ROS 2 tools to find the package. Inside the directory there is \texttt{\twound{init}\twound{.py}}.

In our case, LGSVL team provides the base package. So we need to just build it using
\texttt{colcon build --symlink-install} command. But before building, the ROS2 environment
 must be set. It is important to remember that the package has
to be built every time a new ROS or custom message data types are introduced. \texttt{build},
\texttt{install} and \texttt{log} directories are created along side \texttt{src}
directory when the build command is executed at the parent workspace directory. And
every time before running the package, its local environment must be set. Otherwise,
custom message data types won't be initialised.

\subsection{Using docker-compose services}
So everything that involves ROS starts with setting the environment globally or locally.
It is easy to miss this small step and encounter problems that could take a long time to
resolve. A \texttt{docker-compose} helps alleviate this problem. A service for each task
is implemented such as build, collect and evaluate. In the yaml file, each service has a
keyword to invoke the service and also an argument to link a file. In this case, a
shell script file is created. It contains all the necessary steps
such as setting the environment, starting the package, establishing ROS web bridge etc.
\fi

\begin{figure}
	\centering
    \def\svgwidth{0.6\textwidth}
    \input{figures/inkscape/preprocessing.pdf_tex} %use full path to know the location of pdftex
    \caption{Preprocessing module}
    \label{fig:preprocessing}
\end{figure}

\newpage
\section{Preprocessing}
The stored data can't be always used directly for training. Most times it must be
preprocessed to user's needs and goals.
Inputs are usually represented as \texttt{X\_{data}} and outputs as \texttt{Y\_{data}}. In our case, the input is images and output control commands.
Since we are doing supervised learning, we are aware about the outputs. These are stored in
\textit{csv} files along with file names of the images.

So, the first task in preprocessing is to select which \texttt{Y\oneund{data}} is necessary for prediction
and seperate them out into a small text file. Using this file, the images are fetched,
manipulated using CV2 libraries, stored in arrays and saved in the form of HDF5 files
\cite{hdf5file}. The Hierarchical Data Format version 5 (HDF5), is an open source file format
that supports large, complex, heterogeneous data. Within one HDF5 file, you can store a similar set of data organized in the same way that you might organize files and folders on your computer.
It is a compressed format and supports \textit{data slicing} which allows only a part of
the dataset to be read and not load all of them in the RAM memory.

The images in our case are read either as grayscale or RGB colour images. Then are cropped
and resized to a smaller resolution such as 160x70. For grayscale image there is one
channel. So the image's dimensions resemble 160x70x1 and for RGB image it has 3 channels
which means the dimensions are 160x70x3.

The images from multiple viewpoints or sensors can be fused together making
multi-channels. This task will be explained more in data fusion
section(\ref{sec:datafusion1}).

\subsection{LSTM}


LSTM comprises of serially lined up LSTM cells which allow prediction using previous
data. Since previous data require data from past, each frame image must be backtracked to
a certain, defined time period. This is called \textit{time steps}. According to the time
step, the images(frames) are gathered as one and stored. So for a $time\_step = 15$, the
dimensions will look like \texttt{15x70x160x1} for grayscale images and \texttt{15x70x160x3} for RGB
images.
\begin{figure}[h]
	\centering
    \def\svgwidth{0.8\textwidth}
    \input{figures/inkscape/slidingwindow.pdf_tex} %use full path to know the location of pdftex
    \caption{Sliding frame window implementation module}
    \label{fig:slidingwindow}
\end{figure}

\section{Datafusion}
\label{sec:datafusion1}
Data fusion is one of the primary goals of this thesis. As discussed in fundamentals
chapter(\ref{sec:datafusion}), there are two techniques for data fusion -- early and late
fusion. For early fusion, the images from multiple viewpoints or sensors are fused in the
preprocessing stage. This fusion is accomplished either by stacking the images or
concatenating them. So for example, if a grayscale and RGB images are fused/overlayed together using
concatenation, then the dimensions would like \texttt{70x160x4} where 4 represents number
of channels. These images are usually referred to as \textit{multispectral images}. The figure \ref{fig:cnnarchitecture} illustrates this approach.

Late fusion on the other hand is done during the training stage of the end-to-end work
flow. Usual process involves combining(concatenating) two sources of information after one
or two layers of convolution and then using the combined block to do further feature
extraction and eventually prediction. Or if the source is of a different modality than an
image, then it is unnecessary to fuse them in convolution stage. It is added after
the CNN is completed. However, it must be remembered that late fusion increases the
trainable parameters and costs on resources. The figure \ref{fig:latefusion} illustrates
one of the late fusion processes.

\begin{figure}[h]
    \centering
    \def\svgwidth{\textwidth}
    \input{figures/inkscape/latefusion1.pdf_tex}
    \caption{Late Fusion}
    \label{fig:latefusion}
\end{figure}

\section{Training the model}
\begin{wrapfigure}{l}{0.4\textwidth}
	\centering
    \def\svgwidth{0.4\textwidth}
    \input{figures/inkscape/splitdata.pdf_tex} %use full path to know the location of pdftex
    \caption{Splitting the dataset into train and test data using Sci-kit learn module.}
    \label{fig:splitdata}
\end{wrapfigure}
Training a model involves designing a neural network architecture and deciding on its
hyperparameters. In this thesis, CNN and dense layers are designed with appropriate
activation functions, learning rate, epochs, batch size, CNN specific stride and kernel
lengths, optimizer etc.

\subsection{Loading from HDF5 and splitting the data}
The data stored in HDF5 files in preprocessing are loaded into memory as X\_data and
Y\_data respectively. Then using scikit-learn module, the X\_data is then split 80-20
as X\_train and X\_test respectively. Similarly Y\_data as Y\_train and Y\_test respectively.

\subsection{CNN and fully connected layers}
For CNN layers, feature maps starting from 24 channels is chosen and gradually increased till 64.
The stride is always kept at 2 whereas the kernel size is (5,5) for the early and
(3,3) for the later stages. For early data fusion, the input is already fused and directly
fed to the neural network. However, for late fusion, concatenation is done at appropriate
stages. If necessary, max pooling and batch normalization layers are added to the neural
network. Most often to distribute the features uniformly and to make the cost function
distribute symmetrically, the inputs are normalized. In this case, since images are pixel
values between 0-255, each pixel is divided by 255 to bring it in the range between 0 and
1.
\begin{figure}
    \centering
        \def\svgwidth{1.05\textwidth}
        \input{figures/inkscape/trainingImplementation.pdf_tex} %use full path to know the location of pdftex
    \caption{Implementation of Training module.}
    \label{fig:trainingmodule}
\end{figure}

Since Keras is used, almost all the layers can be implemented in a fewer lines of code.
Activation functions are given as an argument to a layer. Adding new layers is easier with
functional API \ref{subsec:modelsapi}. When the convolutional layers' output needs to be
flattened to form a vector, \texttt{Flatten} command is called.

The fully connected or dense layers take input as a vector. The hyperparameters are
adjusted accordingly to avoid overfitting. Using dropout layers and batch normalization
help alleviate this problem.

Using callbacks functionality of Keras, the best model is saved in HDF5 file. In our case, validation
loss  reaching the minimum is monitored. Since the datasets are not huge, an epoch of 100
is sufficient.

\section{Evaluation}
The trained model is saved as HDF5 file format. Evaluation is basically completes the loop
of end-to-end training architecture. The trained model is placed at a location the
evaluation script can access. Then from the LGSVL simulator data are received through ROS
bridge and subscriber nodes. With the help of message filters, the messages are collected.
Inside the callback, the image manipulation carried out in preprocessing phase, is
repeated. The preprocessed image is then fed to the trained model. The model predicts the
output which in our case, is control commands. These commands are then assigned and published/sent
back to the simulator through ROS bridge. The custom plugin has a subscribing topic on the
LGSVL side. The data sent through ROS bridge, is picked up by nodes listening in this topic. The predicted command behaviour is observed and
evaluated using appropriate metrics. It is important to remember that, the exact steps followed
in preprocessing must be repeated while evaluating. Otherwise, it will lead to inconsistent
performance.

\begin{figure}
	\centering
    \def\svgwidth{1.05\textwidth}
    \input{figures/inkscape/evaluation.pdf_tex} %use full path to know the location of pdftex
    \caption{Evaluation implementation}
    \label{fig:evaluationfigure}
\end{figure}

\iffalse
\section{What to include here?}

\begin{enumerate}
    \item Docker - Dockerfile contents - docker scripts - how they are invoked.
    \item LGSVL - C\# language, Unity engine, code organisation, sensor plugin and its
        structure, data type conversion, WebUI - JSON- sensors used. Making Radar work,
        Increasing traffic density and time at the signal.
    \item Data collection - ROS2- ros2 message types - topics - msgfilters -
        approx-sync - callback- cv2 libraries - saving. Also ROS2 web bridge.
    \item Preprocessing - CV2 image manipulation - stacking, concatenate  -
        Time series LSTM stuff - saving in HDF5.
    \item Training - Loading hdf5 - splitting data - Functional API - normalisation
        - datafusion - early, late - concat - batch normalisation - stride - kernel -
        flatten - FC - prediction - Activation function - loss function - learning rate -
        optimizer - epochs - shuffling - batch size - callbacks - ES - MC - TB - saving
        models.
    \item Evaluation - similar stuff to data collect - image manipulation - prediction -
        publishing. How to validate evaluation?

\end{enumerate}

\fi

\chapter{Evaluation}
In this chapter, the workflow explained in last chapter is evaluated and results are
presented.

Before showing the evaluation, it is necessary to define training and testing conditions
that can be easily used by others to verify the results.
\begin{figure}[h]
    \centering
    \def\svgwidth{0.3\textwidth}
    \input{figures/inkscape/datasets_general.pdf_tex}
    \caption{Datasets distribution}
    \label{fig:datasetsdistribution}
\end{figure}

We have three datasets that can be used for training and evaluation.
\begin{enumerate}
    \item Dataset 1 - Contains 100,000 raw data as seen in figure
        \ref{fig:datasetsdistribution}. It is collected in no traffic
        environment, doing straight driving without any sudden turning. The data is using
        San Francisco map and driven during afternoon. This dataset has only data
        representing centre camera pointed ahead, parallel to the ground and right camera
        pointed to the ground at an angle $20^{\circ}$. The control commands include
        acceleration, throttle, braking, and steering angle values.
    \item Dataset 2 - Also contains 100,000 raw data. It is, however, collected with traffic
        where the cars stop at signal intersections for a longer time than dataset 3. This
        dataset is also collect in San Francisco map and during afternoon. It contains a
        centre camera, right camera like dataset 1, left camera similar to right camera by
        pointing at an angle $20^{\circ}$ to the ground, depth camera sensors placed at
        centre, left and right just like RGB cameras. The control commands are same as
        dataset 1.
    \item Dataset 3 - Contains 270,000 raw data. It is collected while driving around San
        Francisco. About 200,000 data is collected while driving in the afternoon. About
        20,000 in different weather and light conditions. About 50,000 entries are
        collected in a different circular circuit map called CubeTown. In addition to RGB
        and depth cameras distributed just as dataset 2, a segmentation camera is kept
        next to centre RGB camera facing forward, and a radar sensor just in front of the
        car near the hood also facing forward.

\end{enumerate}

\section*{Evaluation setup}
While evaluating, a testing parameter \textit{episode} is used. Each episode lasts 30 seconds. A timer is started for 30 seconds and
the  model is tested for collisions. If a collision happens, the time at which collision
happened is noted.

As supervised learning is used, the models have to be tested/validated with unknown data
to determine its capability. Hence the datasets are split 80-20. Meaning 80\% is train
data and 20\% validation data. The optimizer \textit{Adam} takes the 20\% data to test the
trained model. Training data leads to training loss and test data to validation loss.

Also till a single dataset is chosen, all datasets are of equal data entries.

\section{Determine which datasets and best lighting conditions to test the model}
\label{chapter05subsec:setup1}
All three datasets are used. The test is conducted in San Francisco map without traffic
option switched ON. By varying the light conditions to morning, afternoon and evening, we
observe how light influences the prediction of output(see table \ref{table:timeoftheday}).

The steering angle is a continuous value ranging between -1 and 1(negative values to
turn left and positive values to turn right), a continuous loss function has to be used.
Because of that \textit{mean square error}(MSE) as loss function is chosen.
Only steering angle is predicted
and a steady velocity of 3 meter per second is used. An episode length of 30s is used.
When a collision is observed, the time of collision and the number of collisions are noted down.
\begin{table}[t]
    \centering
\begin{tabular}{cccc}
    \toprule
    time(in 24 hrs standard) & Morning & Afternoon & Evening \\\midrule
      & 7:30 & 15:30 & 18:30 \\\bottomrule
\end{tabular}
\caption{Time of the day}
\label{table:timeoftheday}
\end{table}

\begin{figure}[h]
	\centering
    \def\svgwidth{0.9\textwidth}
    \input{figures/inkscape/lightvscollisionvstraffic.pdf_tex} %use full path to know the location of pdftex
    \caption{a) Datasets vs Light Conditions vs Collisions.
        b) Afternoon - Datasets vs Traffic vs Number of Collisions.
    c) Average number of collisions in percentage}
    \label{fig:dsvslcvstrafficAll}
\end{figure}

It is seen from figure \ref{fig:dsvslcvstrafficAll}(a) that afternoon time provides the best light conditions for all the three
datasets. Dataset 1 and 3 perform equally across the three lighting conditions.

If the percentage of number of collisions with traffic toggled ON, as shown in figure
\ref{fig:dsvslcvstrafficAll}(c), is calculated, dataset 3 performs the best among
the datasets for morning and afternoon part of the day.

\subsection{Datasets performance during afternoon if traffic is enabled}
All three datasets are again used. The time is fixed at 15:30. The traffic is toggled ON.
From figure \ref{fig:dsvslcvstrafficAll}(b), we can observe that all three datasets do
well even in traffic. However, it is surprising to see dataset 1 which had no traffic
while the dataset was collected, performs remarkably well when driven in traffic.
\subsection{Observations}
\begin{enumerate}
    \item All 3 datasets do well at afternoon time of the day.
    \item Predicting only steering angle with MSE as loss function works as seen from
        \ref{fig:dsvslcvstrafficAll}.
    \item Dataset 2 shows higher number of collisions. So it is better to avoid for
        further analysis.
\end{enumerate}

\section{Acceleration - Determine which activation and loss functions to use}

\subsection{Tanh as activation and MSE as loss functions}
Since acceleration and steering values in LGSVL range from \textit{-1} to \textit{1},
\textit{tanh} activation function is selected as the output dense layer activation
function.

A set of criteria are listed and the trained model is evaluated based on these conditions.
From table \ref{table:tanhmse} it can be observed that dataset 1 outperforms dataset 3 in
most of the conditions. Dataset 1 retains good steering control at high speeds and
turning, but acceleration skews steering angle when traffic is switched ON. Dataset 3 when
evaluated stops completely after moving a few metres. This causes difficulty in evaluating
according to the criteria. Hence it is assumed that this dataset fails to meet the
criteria.

One of the reasons as to why dataset 3 fails could be because the losses are much higher
than dataset 1 and the validation loss starts to overfit too quickly in the training(see
figure \ref{fig:ds1andd3tanhactivatonMSE}).
\begin{table}[h]
    \centering
\begin{tabular}{lcc}
    \toprule
    Criteria(Tanh/MSE) & Dataset 1 & Dataset 3 \\\midrule
    Lane keeping/Drive straight  & Yes & No  \\
    Gradual acceleration increase & Yes & No\\
    Smooth braking behaviour observed & Yes & No \\
    Smooth steering control at high speed(10m/s) & Yes & No \\
    Smooth steering control at turnings & Yes & No\\
    Detects traffic as dynamic objects & Yes & No\\
    Navigates traffic smoothly & No & No\\
    Doesn't stop at random places & No & No \\\bottomrule
\end{tabular}
\caption{Tanh/MSE - How the model evaluates to different criteria}
\label{table:tanhmse}
\end{table}
\begin{figure}[h]
	\centering
    \def\svgwidth{\textwidth}
    \input{figures/inkscape/regressionModelsTanhActivation.pdf_tex} %use full path to know the location of pdftex
    \caption{Datasets 1 vs 3 - Acceleration and Steering using Tanh activation and MSE loss
    functions.}
    \label{fig:ds1andd3tanhactivatonMSE}
\end{figure}

\subsection{Sigmoid as activation and MSE as loss functions}
The acceleration values are split into positive and negative values. Instead of negative
values an another variable we will call as \textit{braking} is introduced. Negative
acceleration values mean braking is active. Using this knowledge, sigmoid as activation function and mean
square error as loss function, a training is conducted for both datasets 1 and 3.

Criteria similar to table \ref{table:tanhmse} are put to test with this activation and
loss function. As seen in table \ref{table:sigmse}, both datasets fail to meet the
conditions. During evaluation, both datasets trained models, accelerate to a huge velocity
such as 40m/s and the steering cannot keep up, resulting in collisions.

Looking into their losses (see figure \ref{fig:ds1andd3SigactivatonMSE}), don't really
explain much. Though this needs more investigation, for now we assume this activation or
loss function is not suitable.
\ref{fig:ds1andd3tanhactivatonMSE}.
\begin{table}[h]
    \centering
\begin{tabular}{lcc}
    \toprule
    Criteria(Sigmoid/MSE) & Dataset 1 & Dataset 3 \\\midrule
    Lane keeping/Drive straight  & No & No  \\
    Gradual acceleration increase & No & No\\
    Smooth braking behaviour observed & No & No \\
    Smooth steering control at high speed(10m/s) & No & No \\
    Smooth steering control at turnings & No & No\\
    Detects traffic as dynamic objects & No & No\\
    Navigates traffic smoothly & No & No\\
    Doesn't stop at random places & No & No \\\bottomrule
\end{tabular}
\caption{Sigmoid/MSE - How the model evaluates to different criteria}
\label{table:sigmse}
\end{table}

\begin{figure}[h]
	\centering
    \def\svgwidth{\textwidth}
    \input{figures/inkscape/regressionModelsSigActivation.pdf_tex}
    \caption{Datasets 1 vs 3 - Acceleration and Steering using Sigmoid activation and MSE loss
    functions.}
    \label{fig:ds1andd3SigactivatonMSE}
\end{figure}

\subsection{Softmax as activation and Binary crossentropy as loss functions}
Both tanh and sigmoid activation functions couldn't give stable results to continue
pursuing with those parameters. Hence it is necessary to consider other functions which
may suit our needs.

Since acceleration are basically two discrete values, it would be worthy to try the
training as classification task. The goal of a classification task model is to classify to
which category the prediction belongs to. In our case, acceleration or braking. Hence
\textit{softmax} activation function is needed. As loss function \textit{binary
crossentropy} is used to classify as binary classes. Of course for steering angle,
being continuous, MSE is preferred.

From table \ref{table:softmaxandbce}, both datasets don't do well. Dataset 1 doesn't start
at all as braking class dominates the evaluation whereas dataset 3 starts to move the car
but resorts to brake indefinitely after a few metres of driving. This behaviour
necessitates further analysis into the datasets.
\begin{table}[h]
    \centering
\begin{tabular}{lcc}
    \toprule
    Criteria(Softmax/Binary crossentropy) & Dataset 1 & Dataset 3 \\\midrule
    Lane keeping/Drive straight  & No & Yes  \\
    Gradual acceleration increase & No & No\\
    Smooth braking behaviour observed & No & No \\
    Smooth steering control at high speed(10m/s) & No & No \\
    Smooth steering control at turnings & No & No\\
    Detects traffic as dynamic objects & No & No\\
    Navigates traffic smoothly & No & No\\
    Doesn't stop at random places & No & No \\\bottomrule
\end{tabular}
\caption{Softmax/Binary crossentropy - How the model evaluates to different criteria}
\label{table:softmaxandbce}
\end{table}
\iffalse
\begin{figure}[h]
	\centering
    \def\svgwidth{0.8\textwidth}
    \input{figures/inkscape/BinaryCross2.pdf_tex} %use full path to know the location of pdftex
    \caption{Dataset 1 - Binary Crossentropy}
    \label{fig:ds1binarycrossentropy}
\end{figure}

\begin{figure}[h]
	\centering
    \def\svgwidth{0.8\textwidth}
    \input{figures/inkscape/BinaryCross2ds3.pdf_tex} %use full path to know the location of pdftex
    \caption{Dataset 3 - Binary Crossentropy}
    \label{fig:ds3binarycrossentropy}
\end{figure}
\fi
\subsection*{Control commands distribution}
When the datasets are analysed for patterns of different states -- acceleration and
braking, distribution chart(figure \ref{fig:datasetscomparectrlcmds}) reveals that in addition to
acceleration and braking states, there is a third state called \textit{no action} where
the vehicle does absolutely nothing. In fact this state dominates in both datasets.
Because of this, the binary crossentropy obviously fails to meet the conditions.
\begin{figure}[!ht]
    \centering
    \def\svgwidth{\textwidth}
    \input{figures/inkscape/datasets_control_cmds.pdf_tex}
    \caption{Datasets 1 vs 3 control commands distribution}
    \label{fig:datasetscomparectrlcmds}
\end{figure}

\subsection{Softmax as activation and Categorical crossentropy loss functions}
So now we know that this dominant state \textit{no action} needs a separate label if
classification task has to be continued. After creating the label, we would then have
three labels -- acceleration, braking and no action. Hence, we use a new classification
loss function called \textit{categorical crossentropy}. This loss function classifies
model into each category.

\begin{table}[h]
    \centering
\begin{tabular}{lcc}
    \toprule
    Criteria(Softmax/Categorical crossentropy) & Dataset 1 & Dataset 3 \\\midrule
    Lane keeping/Drive straight  & No & Yes  \\
    Gradual acceleration increase & Yes & Yes\\
    Smooth braking behaviour observed & No & Yes \\
    Smooth steering control at high speed(10m/s) & No & No \\
    Smooth steering control at turnings & No & No\\
    Avoids colliding into static objects & No & No \\
    Detects traffic as dynamic objects & Yes & Yes\\
    Navigates traffic smoothly & No & No\\
    Doesn't stop at random places & No & No \\\bottomrule
\end{tabular}
\caption{Softmax/Categorical crossentropy - How the model evaluates to different criteria}
\label{table:softmaxandcce}
\end{table}

From table \ref{table:softmaxandcce}, dataset 1 fails in most conditions and dataset 3
though performs well in 4 out of 9 conditions, shows bad steering behaviour at traffic or high
speeds. If the acceleration is controlled manually, the model reacts better and adapts
itself.

Since no action state dominates, it is necessary to continue as a classification task.
Also dataset 1 has only a small portion for acceleration and even smaller for
braking. Accordingly, dataset 3 is collected with an attempt to increase acceleration
and braking values share. However, since the vehicle most times has to drive straight, \textit{no action} state even dominates in dataset 3.

The important condition in a classification task is to have balanced classes.
Unfortunately in our case, this balance is not achieved. With this limitation, the
evaluation is carried forward.

\subsection{Observations}
\begin{enumerate}
    \item Steering angle uses tanh activation and MSE loss functions
    \item Classify acceleration prediction as classification task which means softmax as
        activation.
    \item Since acceleration carries three states, categorical crossentropy as loss
        function.
    \item Dataset 3 has more of acceleration and braking states than dataset 1. So dataset
        3 is preferred.
\end{enumerate}
\section{Predicting acceleration - categorical crossentropy}
\label{chapter5sec:cce}
Now that the basic criteria for training is fixed such as which dataset, activation, and
loss functions, we can move ahead and optimise the predicted classes by tuning the
neutral network.

\subsection*{LSTM vs Non-LSTM}
Before going into tuning the neural network, it is necessary to tell that acceleration
prediction needs temporal information; meaning decision to drive slower or faster depends
on the previous, historical frames. LSTM is used for this purpose. When non-LSTM model is used
to predict acceleration, it only predicts for the current frame(doesn't provide past
frames information) which often results in vehicle being stationary.
\begin{figure}[!ht]
    \centering
    \def\svgwidth{0.8\textwidth}
    \input{figures/inkscape/lstmvsnolstm.pdf_tex}
    \caption{Datasets 3 - No LSTM vs LSTM comparison}
    \label{fig:ds3nolstmvslstm}
\end{figure}
For our setup, we choose a $timestep = 15$. That means acceleration of current time frame
is predicted using previous 14 time frames.
\subsubsection*{Determining the optimal LSTM output units}
In Keras, the LSTM units refer to the dimension of hidden state vector \textit{h} that is the state output from RNN cell.
The table \ref{table:unitsvstime}, compares different unit values and the number of
trainable parameters(weight that can be trained during backpropagation) at this LSTM
layer.
In our case, it means that for a time series of 15, there will be 15 \textit{cell states},
15 \textit{hidden states}, and 15 \textit{outputs} each of vector size defined by the
units in the table such as 20, 60 or 100.

Upon evaluation with these different units, 100 output units though has the highest
trainable parameters for this layer alone, retains more information needed for training
the model.  Hence, a LSTM unit of 100 is chosen.
\begin{table}[h]
    \centering
\begin{tabular}{ccc}
    \toprule
    LSTM Output Units & Trainable Parameters(ca.) & Processing time needed \\\midrule
    20 & 20000 & 1hr 44m  \\
    60 & 61000 & 1hr 42m \\
    100 & 434000  & 1hr 40m \\\bottomrule
\end{tabular}
\caption{LSTM Output Units vs Trainable Parameters vs Training time}
\label{table:unitsvstime}
\end{table}
\newpage
\subsection{Basic Model}
\begin{wrapfigure}{l}{0.4\textwidth}
	\centering
    \def\svgwidth{0.4\textwidth}
    \input{figures/inkscape/steeringbasicmodel.pdf_tex} %use full path to know the location of pdftex
    \caption{Basic model}
    \label{fig:steeringbasicmodel}
\end{wrapfigure}

For training, a model as shown in figure \ref{fig:steeringbasicmodel}, is designed and its
result is seen in figure \ref{fig:ds3categoricalcrossentropybasic}. Interestingly, the
training loss curve \textit{follows} the classification loss as it dominates the model. Steering
loss however, after epoch 32 starts to increase. Sure enough, upon evaluation,
the steering was all over the place and acceleration was not stable at all, resulting in
many collisions as shown in table \ref{table:softmaxandcce} (dataset 3 column).

The cause for this behaviour is investigated and it is found that different losses have
different magnitudes. By forcing the neural network to learn/train both classification and
regression losses from a same dense layer causes instability in learning.

\begin{figure}[t]
	\centering
    \def\svgwidth{0.8\textwidth}
    \input{figures/inkscape/categoricalcrossds3.pdf_tex} %use full path to know the location of pdftex
    \caption{Basic model}
    \label{fig:ds3categoricalcrossentropybasic}
    \vspace*{-0.30in}
\end{figure}
\vspace{-.2in}
\subsection{Splitting at the dense layers}
To alleviate some of the burden the second dense layer(dense 2) is split into two separate dense
layers; one for classification outputs and other for steering as show in figure
\ref{fig:steeringdensesplit}. The result \ref{fig:ds3categoricalcrossentropydense}, stops
the strange steering loss increase. Upon evaluation, this model shows better steering
control but still the acceleration is not stable or consistent as shown in table
\ref{table:ccedense}.
\begin{figure}[!ht]
	\centering
    \def\svgwidth{0.5\textwidth}
    \input{figures/inkscape/steeringdensesplit.pdf_tex} %use full path to know the location of pdftex
    \caption{Split at the second dense layer}
    \label{fig:steeringdensesplit}
\end{figure}

\begin{figure}[!ht]
	\centering
    \def\svgwidth{0.8\textwidth}
    \input{figures/inkscape/categoricalcrossds3dense.pdf_tex} %use full path to know the location of pdftex
    \caption{Separate dense layers for classification and steering}
    \label{fig:ds3categoricalcrossentropydense}
\end{figure}

\begin{table}[h]
    \centering
\begin{tabular}{lc}
    \toprule
    Criteria(Softmax/Categorical crossentropy)  & Dataset 3 \\\midrule
    Lane keeping/Drive straight  & Yes  \\
    Gradual acceleration increase  & Yes\\
    Smooth braking behaviour observed & Yes \\
    Smooth steering control at high speed(10m/s) & No \\
    Smooth steering control at turnings & No\\
    Avoids colliding into static objects & No \\
    Detects vehicles as dynamic objects & Yes \\
    Navigates traffic smoothly & No\\
    Doesn't stop at random places & No \\\bottomrule
\end{tabular}
\caption{Separate dense layers - How the model evaluates to different criteria}
\label{table:ccedense}
\end{table}
\newpage
\subsection{Splitting at the LSTM layers}
Continuing the theme of tuning the network, the model is split further at LSTM layer as shown in figure
\ref{fig:steeringlstmsplit}. The main aim here is to see if steering control improves and
is stable for considerable acceleration prediction. From figure
\ref{fig:ds3categoricalcrossentropylstm}, the steering loss gets a marginal gain. Still at
evaluation, the trained models stops at random places, and steering control is not stable at
higher acceleration predicted values. Hence the predicted acceleration value is reduced by
50-70\% and fed to the controller. Sure enough the vehicle exhibits stable movements as
seen from table \ref{table:cceLSTM}.

\begin{figure}[!ht]
	\centering
    \def\svgwidth{0.5\textwidth}
    \input{figures/inkscape/steeringlstmsplit.pdf_tex} %use full path to know the location of pdftex
    \caption{Split at the LSTM layer}
    \label{fig:steeringlstmsplit}
\end{figure}

\begin{figure}[!ht]
	\centering
    \def\svgwidth{0.8\textwidth}
    \input{figures/inkscape/categoricalcrossds3lstm.pdf_tex} %use full path to know the location of pdftex
    \caption{Separate LSTM layers for classification and steering}
    \label{fig:ds3categoricalcrossentropylstm}
\end{figure}
\begin{table}[!ht]
    \centering
\begin{tabular}{lc}
    \toprule
    Criteria(Softmax/Categorical crossentropy)  & Dataset 3 \\\midrule
    Lane keeping/Drive straight  & Yes  \\
    Gradual acceleration increase  & Yes\\
    Smooth braking behaviour observed & Yes \\
    Smooth steering control at high speed(10m/s) & Yes \\
    Smooth steering control at turnings & No\\
    Avoids colliding with static objects & No \\
    Detects vehicles as dynamic objects & Yes \\
    Navigates traffic smoothly & Yes\\
    Doesn't stop at random places(negative case) & Yes \\
    Smooth evaluation experience & No \\\bottomrule

\end{tabular}
\caption{Split at the LSTM layer - How the model evaluates to different criteria}
\label{table:cceLSTM}
\end{table}

\subsection{Using two different NN for acceleration and
Steering}
It can be deduced that optimised weights play an important role on how it influences
steering and acceleration prediction. The model is now split as two different neural
networks(NN) as seen in figure \ref{fig:steeringnnsplit}. Though the result
\ref{fig:ds3categoricalcrossentropy2nn} looks similar to
\ref{fig:ds3categoricalcrossentropylstm}, the model predicts stable, consistent
acceleration values.

From table \ref{table:cce2NN}, it even exhibits occasional turning behaviours at junctions. When it
is exposed to traffic, the model does well to navigate, brake, and accelerate.
\begin{figure}[!ht]
	\centering
    \def\svgwidth{0.5\textwidth}
    \input{figures/inkscape/steeringNNsplit.pdf_tex} %use full path to know the location of pdftex
    \caption{Separate NN training model}
    \label{fig:steeringnnsplit}
\end{figure}

\begin{figure}[!ht]
	\centering
    \def\svgwidth{0.8\textwidth}
    \input{figures/inkscape/categoricalcrossds32nn.pdf_tex} %use full path to know the location of pdftex
    \caption{Separate neural network for Classification and Steering}
    \label{fig:ds3categoricalcrossentropy2nn}
\end{figure}
\begin{table}[!ht]
    \centering
\begin{tabular}{lc}
    \toprule
    Criteria(Softmax/Categorical crossentropy)  & Dataset 3 \\\midrule
    Lane keeping/Drive straight  & Yes  \\
    Gradual acceleration increase  & Yes\\
    Smooth braking behaviour observed & Yes \\
    Smooth steering control at high speed(10m/s) & Yes \\
    Smooth steering control at turnings & No\\
    Avoids colliding with static objects & Yes \\
    Detects vehicles as dynamic objects & Yes \\
    Navigates traffic smoothly & Yes\\
    Doesn't stop at random places(negative case) & Yes \\
    Smooth evaluation experience & Yes \\\bottomrule
\end{tabular}
\caption{Separate neural network for classification and steering outputs - How the model evaluates to different criteria}
\label{table:cce2NN}
\end{table}

A quick overview of steering losses across different NN changes is shown in figure
 \ref{fig:ds3categoricalcrossentropysteeringcompare}.
 \begin{figure}[!ht]
	\centering
    \def\svgwidth{\textwidth}
    \input{figures/inkscape/categoricalcrossds3steeringCompare.pdf_tex} %use full path to know the location of pdftex
    \caption{Steering command loss comparison}
    \label{fig:ds3categoricalcrossentropysteeringcompare}
\end{figure}
\newpage \vfill
\subsection{Observations}
\begin{enumerate}
    \item Tuning the neural network albeit only the fully connected layers, results in
        optimised prediction of steering control corresponding to the classification
        outputs.
    \item The car exhibits random, unknown stops at random places. The actual reason
        behind it is unknown but it is suspected that since the dataset has imbalanced
        classes, it contributes to this random decisions.
    \item Separating into two NNs allows better steering control at a higher speed.
    \item Sunlight and shadows still play a major role. They do some random, strange
        things to models that eventually lead to crashes out-of-nowhere. It could be
        deduced that some buildings' shadows could be considered as static or dynamic
        objects.
\end{enumerate}
\section{Velocity}
Velocity is a scalar value, labelled output. It is considered as an \textit{auxiliary
task}. An auxiliary task usually consists of estimating quantities that are relevant to
solving main supervised learning problem. That means that velocity is predicted as an
auxiliary task using existing CNN-LSTM-Dense architectures without affecting the major
tasks i.e., predicting acceleration and steering.

The images are fed into the models shown in figure \ref{fig:velocitycompareNN} and results are
compared as how predicting velocity affects acceleration and steering.
\begin{figure}[!ht]
    \def\svgwidth{1.15\textwidth}
    \input{figures/inkscape/velocitycompareNN.pdf_tex} %use full path to know the location of pdftex
    \caption{Different architectures used while predicting velocity}
    \label{fig:velocitycompareNN}
\end{figure}

From the figure \ref{fig:velocitycompareloss1}, looking at the loss-epoch graphs of model
\textit{a}, \textit{b}, \textit{c}, the validation losses follow velocity's validation loss.
When compared to classification and steering loss, this loss is too high. Sure enough
while evaluating steering and acceleration are worse. Since velocity is an auxiliary task,
there is a need to rethink how cost function is calculated.
\begin{figure}[!ht]
	\centering
    \def\svgwidth{\textwidth}
    \input{figures/inkscape/velocitycomparelosses5.pdf_tex} %use full path to know the location of pdftex
    \caption{Comparison of losses for NN architectures shown in figure \ref{fig:velocitycompareNN}}
    \label{fig:velocitycompareloss1}
\end{figure}

\subsection{Weighted loss function}
Since velocity is taking control of the neural network to provide feature extraction and
decision making for its prediction, both acceleration and steering suffer bad
consequences. In order to avoid this, weighted cost function is introduced to the three
outputs. As velocity is an auxiliary task, it is suppressed more than others. From figures
\ref{fig:velocitycompareNN} and \ref{fig:velocitycompareloss1}, model \textit{a} gives
nearly optimum losses. So that model is chosen to carry out this weighted cost function
experiment.

The model is tweaked only to include custom, weighted cost function. The trained model
exhibits now losses similar to classification losses than velocity(as seen in figure
\ref{fig:velocitycompareloss1}, \textit{weighted} graph). Indeed during
evaluation, acceleration and steering both get preference and control than velocity.
Hence it is possible to include auxiliary tasks to normal NNs without compromising the
actual functionality of it.

\section{Convolution layers manipulation}
The fully connected/dense layers are tweaked to various designs as shown in
figures \ref{fig:steeringbasicmodel}, \ref{fig:steeringdensesplit},
\ref{fig:steeringlstmsplit}, \ref{fig:steeringnnsplit} and having convolutional layers as
constant. Some interesting possibilities and observations are made possible. As a
consequence, the convolutional layers are changed keeping dense layers as constant.
We are going to make prediction of velocity also. Hence
\ref{fig:velocitycompareNN}(\textit{a}
model) does
relatively well in both setup, convolutional layer experiments are done using this model.

\subsection{Adjusting the width of the convolutional layers}
\subsubsection*{Changing the feature maps channel depth}}
The convolutional layers consists of feature maps channels, kernel filter which convolves
on the input using a specified stride. In our case, the last convolutional layer's feature
map channel's depth is increased from 64 to 80. This after \textit{flatten} layer
increases the trainable parameters. This allows more features to carried into the fully connected layer.
\begin{figure}[!ht]
	\centering
    \def\svgwidth{\textwidth}
    \input{figures/inkscape/velocitysplitconvlayerschange6.pdf_tex} %use full path to know the location of pdftex
    \caption{Convolutional layers width changes - Increasing feature maps channel depth}
    \label{fig:convlayerschange1}
\end{figure}
\subsubsection*{Changing the stride}
Stride is a component in CNN tuned for compressing the images data. This parameter
specifies the movement step of the kernel/filter. In our case, the stride of the 5th
convolutional layer is decreased from (2,2) to (1,1). Hence ensures the information
tensors are not reduced by half and increasing the trainable parameters of the features.

\subsection{Depth of the Convolutional layers}
\subsubsection*{Increase the convolutional layers to eleven}
\begin{figure}[!ht]
	\centering
    \def\svgwidth{1.15\textwidth}
    \input{figures/inkscape/velocitysplitconvlayerschange5.pdf_tex} %use full path to know the location of pdftex
    \caption{Convolutional layers depth changes - Increasing the number of CNN layers}
    \label{fig:convlayerschange3}
\end{figure}
\subsubsection*{Increase the convolutional layers to eight}

\iffalse
\begin{figure}[!ht]
	\centering
    \def\svgwidth{\textwidth}
    \input{figures/inkscape/carsensors.pdf_tex} %use full path to know the location of pdftex
    \caption{Sensor Constellation}
    \label{fig:simplesensorconstellation}
\end{figure}
\section{Testbed setup}

\begin{enumerate}
    \item dataset constant - Only dry day data?
        or everything?
    \item CNN parameters variable.
    \item epochs, learning rate, optimizer, loss function constant
    \item activation function constant or variable?
    \item for LSTM - timesteps value?
    \item graph - training, val loss vs epochs?
    \item graph - loss vs learning rate?
    \item graph - timesteps vs loss?
    \item graph - evaluation performance comparison?
    \item graph - Accel, brake, noaction predicted vs ?
    \item graph - brake, steering predicted vs ?
    \item graph - accel, brake, steering predicted vs ?
    \item graph - accel,brake, steering, distance(radar) predicted vs ?
    \item graph - with seg camera vs without?
    \item graph - with radar vs without?
    \item graph - imbalanced vs balanced Cross entropy?

\end{enumerate}
\fi

%\include{chapter06_}
%\include{chapter07_}

\begin{appendix}

\listoffigures
%\listoftables
%\listoflistings

\raggedright
\bibliographystyle{IEEEtran}
\bibliography{IEEEabrv,bib/IEEEtranBST,bib/mybib.bib}

\end{appendix}

\end{document}


%%%%%%%%%%%%%%%%%%%%%%%%%%%%%%%%%%%%%%%%%%%%%%%%%%%%%%%%%%%%%%%%%%%%%%%%%%%%%%%
%                          END OF THE DOCUMENT                                %
%%%%%%%%%%%%%%%%%%%%%%%%%%%%%%%%%%%%%%%%%%%%%%%%%%%%%%%%%%%%%%%%%%%%%%%%%%%%%%%
