\chapter{Future Work}
One of the motivations of this thesis is to see whether it is possible to use LGSVL
simulator to implement an end-to-end network to predict control commands by using
supervised learning. The results prove that the controlling the car using predicted
commands just by using images is possible.
However, once the velocity of the car goes beyond 3 meters per second, the steering
control becomes erratic. Though the acceleration can be clipped by a percentage, it does
not provide an optimal solution. So in future, it could be possible to understand why the
acceleration behaves unevenly.

When doing classification based training model, it is observed that the dataset are
biased to drive straight without doing anything. Acceleration and braking populate only
about 10\% of the whole dataset. Hence, a more balanced dataset would help train more
classification based loss functions.

As it is seen from the evaluation chapter, the largest dataset has only 270,000 entries
and during training the images are resized to smaller dimensions to avoid memory issues.
So for the next phase, more computing power would help to train more robust models.

