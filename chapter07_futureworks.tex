\chapter{Future Work}
One of the motivations of this thesis is to see whether it is possible to use LGSVL
simulator to implement an end-to-end network to predict control commands by using
supervised learning. The results prove that the controlling the car using predicted
commands just by using images is possible. 
There are, however, several drawbacks which in future can be improved.
\begin{enumerate}
    \item Make the dataset more balanced and robust so that the classifier can make unbiased
        decisions and also make the trained model not affected by sunlight.
    \item Dataset 1 did well in traffic even though it was not exposed to one during data
        collection phase. Dataset 3 on the other hand, failed miserably in evaluation when
        performed as a regression task. A combination of these two datasets made it
        possible to achieve better results. So a dataset consisting of holding the lane
        and careful driving would help in achieving overall better performance.
    \item Though dataset 3 had about 270,000 data entries, the computer didn't have the
        necessary computing power or the resources to hold such amount of data. More
        computational power would help make faster, bigger, and robust models.  Almost all of the training were carried out with input image's dimension as (160,70,1). A significant increase in computing resources will help in increasing image dimensions and convolutional layers for better feature extraction.
    \item Auxiliary task as predicting velocity along with acceleration and steering
        overwhelmed the model if not weighted properly. Still velocity prediction was
        worse. Some work could be done to include these tasks efficiently.
    \item Data fusion with depth images didn't function as expected. Future analysis as to why this is happening and how to rectify it would be a good area of research.
\end{enumerate}

