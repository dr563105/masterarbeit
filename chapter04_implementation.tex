\chapter{Implementation}
\label{chapter:implementation}
This chapter will present the implementation of end-to-end network with its extensions.
First we start with docker to set the environment. Then move on to LGSVL and ROS.
From there a closed loop is achieved to collect data, preprocess, introduce neural
network, implement the models, and evaluate the trained model. After achieving the basic results for the
preliminary architecture, sensor fusion techniques are implemented.

\section{Docker}
Docker is an open-source platform for developing, shipping and running applications.
Because docker makes installing applications hardware independent, we use docker for our
tasks.
\begin{wrapfigure}{l}{0.5\textwidth}
	\centering
    \def\svgwidth{0.5\textwidth}
    \input{figures/inkscape/scrot_dockerengine.pdf_tex} %use full path to know the location of pdftex
    \caption{Docker Engine and its functions}
    \label{fig:dockerengine}
\end{wrapfigure}

A docker architecture, as shown in \ref{fig:dockerarchitecure}, consists of client, host
and registry. To make all these components work, docker daemon is necessary. A daemon is a
type of long-running background process. The LGSVL docker image is either pulled from the registry using
\textit{docker pull} command or built using a \textit{dockerfile}. An image is a read-only template with instructions for
creating a docker container. In our case since the image is readily available, so we pull it.

A docker container for each task can be defined. Along with a task, certain other
services may need to be run along with it. \textit{Docker compose} gives a perfect
solution to manage docker applications. As seen in figure \ref{fig:docker1},
docker-compose helps execute multiple commands. Upon execution, ROS environment are set
then, a websocket with a port exposed. In our case, it is \texttt{http://localhost:9090}.
The LGSVL on the other side through its web user interface(WebUI), listens on this port. So a bridge is
established to flow of data.
\begin{figure}[!ht]
    \centering
    \def\svgwidth{0.8\textwidth}
    \input{figures/inkscape/docker_1.pdf_tex} %use full path to know the location of pdftex
    \caption{Docker and its various functions}
    \label{fig:docker1}
\end{figure}

\section{LGSVL simulator}
The LGSVL simulator \footnote{LGSVL simulator version till May 2020 was used for this
    research work. June 2020 presents new changes to how rosbridges are run which would require slight
change to steps of operation.} is developed using Unity engine which is written in C\# language.
The LGSVL team organises their code base\cite{lgsvlgithub} in such a way that it makes it
easy for a beginner to learn the structure and either implement new features or change the
existing ones.

\subsection{Simulator configurations}
The LGSVL team has developed a WebUI to help users to configure maps, vehicles and
simulations. The LGSVL application connects with the WebUI through a
websocket(\texttt{http://localhost:8080}). The users are allowed to configure the
simulation settings using the UI. These configurations are stored in JSON format. Some of the JSON parameters
are parsed in the UI itself and some are transferred to the application using
\textit{http} protocol.

\begin{wrapfigure}{l}{0.4\textwidth}
	\centering
    \def\svgwidth{0.4\textwidth}
    \input{figures/inkscape/webui.pdf_tex} %use full path to know the location of pdftex
    \caption{LGSVL Simulator - WebUI}
    \label{fig:lgsvlwebui}
    \vspace{-15pt}
\end{wrapfigure}

\begin{figure}[!ht]
	\centering
    \def\svgwidth{0.6\textwidth}
    \input{figures/inkscape/LGSVLsoftwarearchi.pdf_tex} %use full path to know the location of pdftex
    \caption{LGSVL software architecture}
    \label{fig:lgsvlswarchitecture}
\end{figure}
Sensor parameters are defined in vehicles \textit{tab} of the WebUI in JSON format. If a user wishes to use a colour camera
sensor, then they need to use the JSON format appropriate for this sensor to the vehicle
configuration. Each sensor has a \textit{topic name}. The LGSVL application parses the JSON
parameters and stores them appropriately. In order for the ROS 2 nodes to subscribe to the
sensors, they would need to use the exact topic name as defined in the vehicles tab.

In our case, we use a variety of sensors:
\begin{enumerate}
    \item RGB colour camera,
    \item Depth camera,
    \item Segmentation camera(uses the output of a RGB camera sensor and internally segments them
according to the tags defined by the user in the sensor parameter),
    \item Radar sensor.
\end{enumerate}


The user is also given the flexibility to arrange/align these sensors in different constellations according to
requirements using a \textit{transform} parameter.
\begin{figure}[!ht]
	\centering
    \def\svgwidth{\textwidth}
    \input{figures/inkscape/carsensors.pdf_tex} %use full path to know the location of pdftex
    \caption{Sensor Constellation}
    \label{fig:simplesensorconstellation}
\end{figure}

So, we have(as seen in fig. \ref{fig:simplesensorconstellation})

\begin{enumerate}

    \item a RGB camera placed facing ahead parallel to the ground, another on
the left and right side of the car pointed an angle towards the ground,
    \item a depth camera following same configuration as RGB,
    \item a segmentation camera placed adjacent to the RGB front facing camera,
    \item a radar sensor placed front of the car near to the hood pointing ahead.
\end{enumerate}

\subsection{Inside LGSVL simulator framework}
The figure \ref{fig:lgsvlswarchitecture} shows some of the major functionalities inside
simulator framework. When the application is first started, the WebUI is called. Natively,
the app connects to UI via a websocket. Maps, vehicles, and simulation settings are
communicated and initialised.

As these settings get transferred, internally, a \textit{simulator manager} program is
triggered. This is a main manager responsible for the overall functionality of the
simulator. From this manager program, other managers are invoked when necessary. The
\textit{sensor classes} is responsible for coordinating with all sensors, the \textit{NPC
manager} manages traffic, signal intersections, speed limits, etc., the \textit{controller
manager} manages controls for vehicle movement. Natively, LGSVL uses Unity's \textit{PhysX} engine
for vehicle dynamics. Using Unity's \textit{HDRP} graphics engine, the simulated environment is visualised. The
\textit{UI and Camera managers} are then responsible for the display of the environment in
the application.

So, the sensor manager that oversees the sensor JSON parameters are initialised and assigned correctly, and the
information from the environment using respective sensors. These sensors' data are then
type cast into appropriate message data type using C\# structures. Now the data resembles
ROS message type. This is then passed on to \textit{bridge classes} to be formatted as JSON
and consequently publish to the subscribing ROS nodes via a bridge.
The topics defined in JSON are used to publish respective sensor data.

\subsubsection*{Control Commands}
The control manager is responsible for vehicle control. The steering control is defined
using Unity math function as continuous float values between $-1$ and $1$. The velocity is
determined by distance travelled in the map over time. Its unit is \textit{meter per
second}. The acceleration, throttle and braking are defined as absolute values and are
interconnected to one another. Acceleration ranges from $-1$ to $1$. While accelerating,
it is set as $1$. Throttle is also assigned this value. While braking, the acceleration is
set $-1$ while braking $1$.
\begin{enumerate}
    \item Steering: ${a^s \in \mathbb{R}|  -1 \leq a^s \leq 1}$
    \item Acceleration: ${a^g \in \mathbb{N}|  -1 \leq a^g \leq 1}$
    \item Throttle: ${a^t \in \mathbb{N}|  0 \leq a^t \leq 1}$
    \item Brake: ${a^b \in \mathbb{N}|  0 \leq a^b \leq 1}$
\end{enumerate}

\iffalse
\subsection{Radar Sensor}
Since data fusion is one of the goals of the thesis, using a radar sensor would provide
important depth information. However, in LGSVL, in its current version, the radar sensor is not working as
required. This necessitates some changes to some of the files in the LGSVL code base.
This process involves -
\begin{enumerate}
    \item Correcting the already existing radar sensor code to detect traffic properly and
        assign the data to their variables that look similar to ROS custom message standards.
    \item Converting the LGSVL data to ROS understandable  custom message formats.
    \item Adding ROS2 as the bridge type to establish between client and LGSVL.
    \item On the client side, editing the docker to include custom radar message types.
\end{enumerate}
\fi
\subsection{Sensor plugin}
When a user doesn't want to disturb the current setup of the simulator but rather wants to add
some custom sensor to the vehicle configuration, sensor plugin can be used.  A custom sensor is useful when there is a need to combine multiple standard ROS
message type or define custom message types. LGSVL allows this functionality. A set of guidelines must be followed while developing the plugin. In our case,
it is necessary to have a sensor plugin that would create a sensor and topics. This sensor
and these topics would then be used to fetch data from the simulator, do appropriate tasks and transfer it
bridge classes for transmission.

This custom plugin as shown in figure \ref{fig:lgsvlsensorplugin} extends the unity engine libraries just like
any other sensors to read the values from the JSON definition. Upon \textit{OnAwake}, the
vehicle dynamics is initialised. Then a check whether a bridge is available is done by
contacting the \textit{bridgefactory} which holds all the bridge classes definition. In
\textit{OnBridgeSetup} method, publisher and subscriber w.r.t to the LGSVL simulator is
created with topic names. At a fixed time interval, \textit{FixedUpdate} method is invoked.
Here the values obtained from vehicle dynamics are assigned appropriately. In addition if
the task is to publish, the values such as steering, throttle, braking, velocity are
converted to ROS defined JSON format through \textit{C\# ROS message structures}.
If the task is subscribing, the data received through the topic such as predicted control values from the
neural network model, are decoded from JSON and assigned to LGSVL variables which is reflected in the application.

\begin{figure}
	\centering
    \def\svgwidth{0.8\textwidth}
    \input{figures/inkscape/sensorplugin.pdf_tex} %use full path to know the location of pdftex
    \caption{Inside sensor plugin}
    \label{fig:lgsvlsensorplugin}
\end{figure}
LGSVL simulator is now configured to send data towards the client. In order to reach the
client, as mentioned before, a rosbridge is needed. In the next section we will talk
about ROS and its uses.

\section{Data Collection Module}
The figure \ref{fig:datacollectionmodule} gives an overview of the data collection task of
sensor module.

\subsection{An overview of data gathering task}
ROS, in our case, acts as an interface between simulator(server) and scripts(client).
We use ROS 2 and in particular \textit{dashing} iteration.
The script's ROS 2 subscribing nodes listen to the sensors' topics(defined using JSON
sensor parameters) and invoke a callback whenever they receive data. Since each sensor receives at different rates, a
filter called message filters is used. With message filters, the queue size is set to a
higher value for example, 1000 and a delay(in seconds) through a \textit{slop} parameter of value
0.1 are used. This filter gathers all the subscribing nodes as one, synchronises
approximately to the delay parameter and invokes just one callback. This assures that data
from each listening node is present.
\begin{figure}
	\centering
    \def\svgwidth{\textwidth}
    \input{figures/inkscape/datacollection.pdf_tex} %use full path to know the location of pdftex
    \caption{A detailed summary of data collection module}
    \label{fig:datacollectionmodule}
\end{figure}

Inside the callback, the ROS 2 sensor data received have a header and data parts.The header part consists of the time at which the message is created and data part
contains the real data. The next step would be to extract the real data. If the data are images then image processing is
done. If not, the scalar values are stored in a CSV file. Then, using numpy libraries the
real data is stored. The images are stored as image files. The non-image/scalar data  along
with corresponding image files(with filenames) are
stored in a CSV file.

\subsection{ROS web bridge}
In the figure \ref{fig:ros2webbridge}, we can see a ROS web bridge is a virtual bridge between
scripts using ROS and LGSVL simulator.
The \textit{ros2-web-bridge} is just \textit{rosbridge} which follows ROSbridge 2.0 protocol
written in nodejs. It basically starts an instance
that listens to an IP address and its port(\texttt{http://localhost:9090}). The LGSVL on the other side(defined inside
WebUI), initiates this bridge on their side at this IP address and port. Hence a bridge is created to allow flow of data.

The message transport is in JSON objects. The rosbridge server is a server which accepts websockets connections and implements the rosbridge protocol.
With the help of rosbridge, data is exchanged with
simulator and ROS2 nodes.

\begin{figure}[h]
	\centering
    \def\svgwidth{0.8\textwidth}
    \input{figures/inkscape/ros2bridge.pdf_tex} %use full path to know the location of pdftex
    \caption{ROS2 web bridge implementation}
    \label{fig:ros2webbridge}
\end{figure}

\iffalse
\subsection{Building ROS2 package}
Before running the scripts with ROS, it must be built as ROS packages.  A package is a
container for ROS 2 code which makes it easier to share with others. Package creation in
ROS 2 uses \textit{ament} as its build system and \textit{colcon} as its build tool.
Packages can be created either in \textit{CMake} or \textit{Python}.

For CMake, \texttt{package.xml} and \texttt{CMakeLists.txt} files are necessary. \texttt{package.xml}
file contains meta information about the package. \texttt{CMakeLists.txt} file describes how to build the code within the package.

For Python, \texttt{setup.py}, \texttt{package.xml}, \texttt{setup.cfg} and
\texttt{resource/<package-name>} are needed. The \texttt{package.xml} file contains meta
information about the package. Unlike CMake, \texttt{setup.py} contains instructions on
how to install the package. The \texttt{setup.cfg} is required when a package has
executables, so \textit{ros2 run} can find them. Lastly there is
\texttt{resource/<package-name>}. It is a directory with the same name as the package,
used by ROS 2 tools to find the package. Inside the directory there is \texttt{\twound{init}\twound{.py}}.

In our case, LGSVL team provides the base package. So we need to just build it using
\texttt{colcon build --symlink-install} command. But before building, the ROS2 environment
 must be set. It is important to remember that the package has
to be built every time a new ROS or custom message data types are introduced. \texttt{build},
\texttt{install} and \texttt{log} directories are created along side \texttt{src}
directory when the build command is executed at the parent workspace directory. And
every time before running the package, its local environment must be set. Otherwise,
custom message data types won't be initialised.

\subsection{Using docker-compose services}
So everything that involves ROS starts with setting the environment globally or locally.
It is easy to miss this small step and encounter problems that could take a long time to
resolve. A \texttt{docker-compose} helps alleviate this problem. A service for each task
is implemented such as build, collect and evaluate. In the yaml file, each service has a
keyword to invoke the service and also an argument to link a file. In this case, a
shell script file is created. It contains all the necessary steps
such as setting the environment, starting the package, establishing ROS web bridge etc.
\fi

\begin{figure}
	\centering
    \def\svgwidth{0.5\textwidth}
    \input{figures/inkscape/preprocessing.pdf_tex} %use full path to know the location of pdftex
    \caption{Preprocessing module}
    \label{fig:preprocessing}
\end{figure}

\section{Training Module}

The stored data can't be always used directly for training. Most times it must be
preprocessed to user's needs and goals. Input is \textit{X\_{data}} and output
\textit{Y\_{data}}. Since we use supervised learning algorithm, the output is known and
labelled.

\subsection{Preprocessing}
As shown in figure \ref{fig:preprocessing}, the first task in preprocessing is to select which non-image sensor \textit{Y\_{data}} is necessary for prediction
and separate them out into a small text file. Using this file, the images which contain
timestamp in the filename are fetched, manipulated using CV2 libraries, stored in arrays and saved in the form of HDF5 files
\cite{hdf5file}. The Hierarchical Data Format version 5 (HDF5), is an open source file format
that supports large, complex, heterogeneous data. Within one HDF5 file, you can store a similar set of data organized in the same way that you might organize files and folders on your computer.
It is a compressed format and supports \textit{data slicing} which allows only a part of
the dataset to be read and not load all of them in the RAM memory.

The images in our case are read either as grayscale or RGB colour images. Then are cropped
and resized to a smaller resolution such as 160x70. For grayscale image there is one
channel. So the image's dimensions resemble 160x70x1 and for RGB image it has 3 channels
which means the dimensions are 160x70x3.

The images from multiple viewpoints or sensors can be fused together making
multi-channels. This task will be explained more in data fusion
section(\ref{chap05Implsubsec:datafusion1}).

\subsection{LSTM}

LSTM comprises of serially lined up LSTM cells which allow prediction using previous
data. Since previous data require data from past, each frame image must be backtracked to
a certain, defined time period. This is called \textit{time steps}. According to the time
step, the images(frames) are gathered as one and stored. So for a $time\_step = 15$, the
dimensions will look like \texttt{15x70x160x1} for grayscale images and \texttt{15x70x160x3} for RGB
images. In the figure \ref{fig:slidingwindow}, the time\_step acts as a frame window. This
window is moved one step to the right for every image file(frame).

So for every frame(image file), its previous 14 frames are stored along with it. However,
for \textit{Y\_{data}} only the current frame's output is stored as we should match the
current frame with its output.

Another feature we use is restricting how big is the gap between two frames. This is
essential because if the consecutive frames are separated by a bigger margin, combining
them for previous may lead to unknown problems. The time period we use is 1s.
\begin{figure}[h]
	\centering
    \def\svgwidth{0.8\textwidth}
    \input{figures/inkscape/slidingwindow.pdf_tex} %use full path to know the location of pdftex
    \caption{Sliding frame window implementation module}
    \label{fig:slidingwindow}
\end{figure}

\subsection{Datafusion}
\label{chap05Implsubsec:datafusion1}
Data fusion is one of the primary goals of this thesis. As discussed in fundamentals
chapter(\ref{sec:datafusion}), data fusion techniques can be broadly classified  into two techniques -- early and late
fusion. For early fusion, the images from multiple viewpoints or sensors are fused in the
preprocessing stage. This fusion is accomplished either by stacking the images or
concatenating them. So for example, if a grayscale and RGB images are fused/overlayed together using
concatenation, then the dimensions would like \texttt{70x160x4} where 4 represents number
of channels. These images are usually referred to as \textit{multispectral images}. The figure \ref{fig:cnnarchitecture} illustrates this approach.

Late fusion on the other hand is done during the training stage of the end-to-end work
flow. Usual process involves combining(concatenating) two sources of information after one
or two layers of convolution and then using the combined block to do further feature
extraction and eventually prediction. Or if the source is of a different modality such as
distance or velocity or number of vehicles in front of the ego car, they don't need to be
feature extracted like imaged based pixel values as they are scalar values and easy of
understand. Hence, it is added after the CNN is completed and the CNN outputs are made
into a vector through flattening process. However, it must be remembered that late fusion increases the
trainable parameters and costs on resources. The figure \ref{fig:latefusion} illustrates
one of the late fusion processes.

\begin{figure}[h]
    \centering
    \def\svgwidth{\textwidth}
    \input{figures/inkscape/latefusion1.pdf_tex}
    \caption{Late Fusion}
    \label{fig:latefusion}
\end{figure}


\subsection{Loading from HDF5 and splitting the dataset}
The data stored in HDF5 files in preprocessing are loaded into memory as \textit{X\_data} and
\textit{Y\_data} respectively. Then using scikit-learn module, the \textit{X\_data} is then split 80-20
as \textit{X\_train} and \textit{X\_test} respectively. Similarly \textit{Y\_data} as \textit{Y\_train} and
\textit{Y\_test} respectively.

\begin{wrapfigure}{l}{0.3\textwidth}
	\centering
    \def\svgwidth{0.3\textwidth}
    \input{figures/inkscape/splitdata.pdf_tex} %use full path to know the location of pdftex
    \caption{Splitting the dataset into train and test data using Sci-kit learn module.}
    \label{fig:splitdata}
\end{wrapfigure}

\subsection{CNN and fully connected layers}

Training a model involves designing a neural network architecture and deciding on its
hyperparameters. In this thesis, CNN and dense layers are designed with appropriate
activation functions, learning rate, epochs, batch size, CNN specific stride and kernel
lengths, optimizer etc.

As shown in the figure \ref{fig:trainingmodule}, the CNN layers extract features from
images through convolution. The features extracted can be stored inside feature maps. Depending on the input image dimensions, the
feature maps' values are adjusted to store as many features as possible.

After extraction, these multi-dimensional data are transformed to 1-D vector through
\textit{flatten} process. Scalar, non-image sensor data can be fused at this stage. The
vector 1-D data are then fed to fully connected layers which are gradually reduced till
the output layer units matches the needed outputs.

The predicted output $\hat{y}$ is compared with the true \textit{Y\_train} output. The
difference, \textit{loss} is then minimised using an optimizer which does the
backpropagation to adjust weights at each layer and node. The best, optimised output model
is then stored in a HDF5 file.
\iffalse
For CNN layers, feature maps starting from 24 channels is chosen and gradually increased till 64.
The stride is always kept at 2 whereas the kernel size is (5,5) for the early and
(3,3) for the later stages. For early data fusion, the input is already fused and directly
fed to the neural network. However, for late fusion, concatenation is done at appropriate
stages. If necessary, max pooling and batch normalization layers are added to the neural
network. Most often to distribute the features uniformly and to make the cost function
distribute symmetrically, the inputs are normalized. In this case, since images are pixel
values between 0-255, each pixel is divided by 255 to bring it in the range between 0 and
1.
\fi
\begin{figure}
    \centering
        \def\svgwidth{\textwidth}
        \input{figures/inkscape/trainingImplementation.pdf_tex} %use full path to know the location of pdftex
    \caption{Implementation of Training module.}
    \label{fig:trainingmodule}
\end{figure}
\iffalse
Since Keras is used, almost all the layers can be implemented in a fewer lines of code.
Activation functions are given as an argument to a layer. Adding new layers is easier with
functional API \ref{subsec:modelsapi}. When the convolutional layers' output needs to be
flattened to form a vector, \texttt{flatten} command is called.

The fully connected or dense layers take input as a vector. The hyperparameters are
adjusted accordingly to avoid overfitting. Using dropout layers and batch normalization
help alleviate this problem.

Using callbacks functionality of Keras, the best model is saved in HDF5 file. In our case, validation
loss  reaching the minimum is monitored. Since the datasets are not huge, an epoch of 100
is sufficient.
\fi
\section{Evaluation Module}
The figure \ref{fig:evaluationfigure} gives an overview of the control module where the
predicted control commands are evaluated.
Using docker-compose the ros2-web-bridge and the evaluation are executed. Then at the
LGSVL end, the same IP address and port are entered and listened in. A rosbridge is
achieved.

Evaluation is basically completes the loop of end-to-end training architecture.  The LGSVL simulator data are received through ROS
bridge and subscriber nodes. With the help of message filters, the messages are collected.
Inside the callback, the sensor data is extracted and image manipulation carried out in preprocessing phase, is
repeated. The preprocessed image is then fed to the trained model. The model predicts the
output which in our case, is control commands. These commands are then assigned and published/sent
back to the simulator(sensor plugin) through rosbridge. The custom sensor plugin has a subscribing topic on the
LGSVL side. The data sent through rosbridge, is picked up by appropriate data type. The predicted command behaviour is observed and
evaluated using appropriate metrics. It is important to remember that, the exact steps followed
in preprocessing must be repeated while evaluating. Otherwise, it will lead to inconsistent
performance.

\begin{figure}
	\centering
    \def\svgwidth{1\textwidth}
    \input{figures/inkscape/evaluation.pdf_tex} %use full path to know the location of pdftex
    \caption{Evaluation Module}
    \label{fig:evaluationfigure}
\end{figure}

\iffalse
\section{What to include here?}

\begin{enumerate}
    \item Docker - Dockerfile contents - docker scripts - how they are invoked.
    \item LGSVL - C\# language, Unity engine, code organisation, sensor plugin and its
        structure, data type conversion, WebUI - JSON- sensors used. Making Radar work,
        Increasing traffic density and time at the signal.
    \item Data collection - ROS2- ros2 message types - topics - msgfilters -
        approx-sync - callback- cv2 libraries - saving. Also ROS2 web bridge.
    \item Preprocessing - CV2 image manipulation - stacking, concatenate  -
        Time series LSTM stuff - saving in HDF5.
    \item Training - Loading hdf5 - splitting data - Functional API - normalisation
        - datafusion - early, late - concat - batch normalisation - stride - kernel -
        flatten - FC - prediction - Activation function - loss function - learning rate -
        optimizer - epochs - shuffling - batch size - callbacks - ES - MC - TB - saving
        models.
    \item Evaluation - similar stuff to data collect - image manipulation - prediction -
        publishing. How to validate evaluation?

\end{enumerate}

\fi
