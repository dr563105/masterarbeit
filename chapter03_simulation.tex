\chapter{Simulation and Simulator}
From the beginning of autonomous driving research, simulators have played a key role in
development and testing new algorithms. Simulators allow developers to quickly test their
algorithms without driving real vehicles. In this chapter, we will the conditions a
simulator must satisfy, and go in detail about LGSVL simulator and its development.

One of the important questions to ask before explaining about simulation is to understand
why should one need a simulator to do simulation. As explained in the previous chapter
\ref{chapter2sec:deeplearning}, deep neural network(DNN) using supervised learning algorithm
needs huge amount of labelled data. Since the cost of collecting that amount of data in
real road vehicle is too expensive, researchers have sought the help of simulators. A
simulator is an application which simulates a real-world environment, virtually.

\section{Conditions for a simulator}

Data collection is one of the most important phases in supervised learning. So caution must
be taken in choosing a simulator for autonomous driving. A simulator must fulfil certain conditions to be
qualified as a good one.
\begin{itemize}
    \item It must have a vehicle that can move around in a virtual map.
    \item The vehicle must be equipped with appropriate sensors for perceiving the
        environment properly.
    \item The virtual map must try to mirror the real-world to an extent. That mean it
        should have proper terrain to drive around, lane marking for lane detection, other
        cars to mirror the real world traffic, pedestrians, and real world weather
        conditions.
    \item It must provide a medium to collect data and allow interfaces to transfer the
        data. It should also be able to receive data in case the user needs to validate
        the data collected.
    \item Finally and most importantly, support end-to-end, full stack simulation.
\end{itemize}

\section{LG SVL simulator}
A simulator chosen for this thesis is from LG research centre in Silicon Valley,
California called LGSVL simulator. It is an open source project where the code is
regularly published at Github \cite{lgsvlgithub}. This simulator satisfies all the
conditions listed above. They provide an out-of-the-box solution which can meet the
needs of developers wishing to focus on testing their autonomous vehicle algorithms. It
also supports Apollo \cite{ApolloAuto} and Autoware \cite{autowarePaper}.

\subsection{LGSVL simulator development}
LGSVL simulator's core simulation engine is developed using the Unity game engine
\cite{unitygameengine}. Unity game engine is written in C\# programming language. Since a
game engine inherently supports animation, the simulator is able to extend that
functionality easily. In addition to Unity, also supports several libraries necessary to
compute complex mathematical operations. With Unity's latest High Definition Render
Pipeline(HDRP), LGSVL is able to simulate photo-realistic virtual environments that match
the real world.

\subsection{Overview of LG SVL simulator}

\subsubsection*{User AD Stack}
It supports user autonomous driving(AD) stack. That means a user can develop, test and verify through simulation.
The user AD stack connects to LGSVL Simulator through a communication bridge interface; a bridge is selected based
on the user AD stack’s runtime framework. This bridge interface can use a standard
protocol such ROS, ROS2 or custom one like CyberRT \cite{ApolloAuto}.

In addition LGSVL supports plug-in component which a user can develop and attach it to the
simulator. The simulator during runtime picks up this plug-in.

\subsubsection*{Simulation Engine}
As mentioned above, LGSVL uses Unity's latest HDRP game engine.

\subsubsection*{Sensor and vehicle models}
It supports sensor arrangement and importantly they are customisable. The sensors are
added and removed through JSON formatted text along with its parameters. These parameters
include sensor type, position of the sensor, topic name, publishing rate, and in some
sensors reference frame of measurement. Some of the popular sensors like camera sensors,
radar and LIDAR are supported. In addition, users can add their own custom sensors as
plug-in. Fig.\ref{fig:sensortypeslgsvlnew} gives a good overview of some of the sensors in action.

Vehicles provide a medium to travel the environment. Hence, vehicle dynamics is also
important.

\begin{figure}[!ht]
    \centering
    \def\svgwidth{0.9\columnwidth}
    \input{figures/inkscape/lgsvlsensors.pdf_tex}
    \caption{Different types of sensors in LGSVL simulator. Anticlockwise(from top): Depth
    camera, LiDAR, Radar(also 3D bounding boxes),and Segmentation camera}
    \label{fig:sensortypeslgsvlnew}
\end{figure}

\subsubsection*{Environment and maps}
An environment, in this case, virtual, is a primary component in autonomous driving
simulation to provide many input to AD system. An environment affects almost all the
functionalities in a AD system such as perception, prediction and tracking modules. It
also affects the vehicle dynamics which is the key factor in vehicle control mechanism.
Through changes in the HD map, the environment affects localization and planning modules.
Finally, weather conditions such as rain, fog, night driving naturally affect the
environment. So caution must be taken while design the environment.

LGSVL supports creating, editing and exporting HD maps of existing 3d virtual environment.
3D environment also defines the rules about how agents must behave such as stopping at traffic
lights, giving way to priority traffic, respect lane boundaries etc.

As of writing, LGSVL supports virtual Sanfranciso city HD map. They also support smaller
maps like Shalun and Cubetown.

\begin{figure}[!ht]
    \centering
    \def\svgwidth{\textwidth}
    \input{figures/inkscape/lgsvlweatherconditions1.pdf_tex}
    \caption{LGSVL simulator in different weather conditions}
    \label{fig:weatherconditions}
\end{figure}

\subsubsection*{Test scenarios}
Test scenarios enable users to test their AD stack by simulating in an environment and
comparing and contrasting correct and expected behaviours. A lot of variables like HD maps,
traffic movement behaviour and their density, time of the day, weather conditions etc. also
play a role while testing. It is also possible to write scripts with the help of Python
API where scenarios can be created and tested.

Thus LGSVL simulator \cite{rong2020lgsvl} provides the best virtual environment to conduct our experiments for
autonomous driving.
